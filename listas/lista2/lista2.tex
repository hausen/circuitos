\documentclass[a4paper,12pt]{article}
\usepackage{graphicx}
\usepackage[left=2cm,top=2cm,bottom=2.5cm,right=2cm]{geometry}
\usepackage[utf8]{inputenc}
\usepackage[T1]{fontenc}
\usepackage{lmodern}
\usepackage[portuguese,brazil]{babel}

\usepackage{amsmath}
\usepackage{slashbox}
\usepackage{array}
\usepackage{icomma} % para vírgula decimal / decimal comma
\usepackage{enumerate}

\input{squarecells.inc}

\def\nt{\overline}

\newcounter{questao}
\setcounter{questao}{0}
\newcommand{\questao}{%
\vspace{12pt}%
\refstepcounter{questao}%
\noindent%
\textbf{Questão \arabic{questao}.}%
{ }%
}

\newcounter{floyd}

\def\N{\vspace{8pt}\addtocounter{floyd}{1}\textbf{\arabic{floyd}.} }

\begin{document}
\begin{center}
\Large{Circuitos Digitais -- Segunda Lista de Exercícios}
\end{center}

\noindent
Observação: o início da lista é composto dos problemas recomendados do livro-texto. Os exercícios nas últimas duas páginas da lista são novos (não estão no livro-texto).

\vspace{12pt}

\noindent
\textbf{Problemas do Capítulo 4 do livro do Floyd.}

\N Usando a notação booleana, escreva uma expressão que é $1$ toda vez que
uma ou mais de suas variáveis $A, B, C, D$ for $1$.

\N Escreva uma expressão que é $1$ somente se todas as variáveis
$A, B, C, D, E$ forem $1$

\N Escreva uma expressão que é $1$ quando uma ou mais das variáveis
$A, B, C$ forem $0$.

\N Avalie as seguintes expressões booleanas:

(a) $0 + 0 + 1$ \;\;\;\; (b) $1+1+1$ \;\;\;\; (c) $1\cdot0\cdot0$ \;\;\;\;
(d) $1\cdot1\cdot1$ \;\;\;\; (e) $1\cdot0\cdot1$ \;\;\;\;
(f) $1\cdot1 + 0\cdot1\cdot1$

\N Encontre os valores das variáveis que fazem com que cada termo-produto
seja $1$ e cada termo-soma seja $0$:

(a) $A \, B$ \;\;\;\; (b) $A \, \nt{B} \, C$ \;\;\;\; (c) $A + B$ \;\;\;\;
(d) $\nt{A} + B + \nt{C}$

(e) $\nt{A} + \nt{B} + C$ \;\;\;\; (f) $\nt{A} + B$ \;\;\;\;
(g) $A \; \nt{B} \; \nt{C}$

\N Encontre o valor de $X$ para todos os valores possíveis para as
variáveis

(a) $X = (A+B)C + B$ \;\;\;\; (b) $X = (\overline{A+B})C$ \;\;\;\;
(c) $X = A\nt{B}C + AB$

(d) $X = (A + B)(\nt{A} + B)$ \;\;\;\; (e) $X = (A + BC)(\nt{B} + \nt{C})$

\N Identifique as regras da álgebra booleana nas quais as seguintes
igualdades se baseiam:

(a) $A \nt{B} + CD + A\nt{C}D + B = B + A\nt{B} + A\nt{C}D + CD$

(b) $AB\nt{C}D + \nt{ABC} = D\nt{C}BA + \nt{CBA}$

(c) $AB(CD + E\nt{F} + GH) = ABCD + ABE\nt{F} + ABGH$

\N Identifique as regras da álgebra booleana nas quais as seguintes
igualdades se baseiam:

(a) $\nt{\nt{AB + CD}} + \nt{EF} = AB + CD + \nt{EF}$ \;\;\;\;
(b) $A\nt{A}B + AB\nt{C} + AB\nt{B} = AB\nt{C}$

(c) $A(BC + BC) + AC = A(BC) + AC$ \;\;\;\;
(d) $AB(C + \nt{C}) + AC = AB + AC$

(e) $A\nt{B} + A\nt{B}C = A \nt{B}$ \;\;\;\;
(f) $ABC + \nt{AB} + \nt{ABC}D = ABC + \nt{AB} + D$

\N Aplique os teoremas de DeMorgan a cada expressão:

(a) $\nt{A + \nt{B}}$ \;\;\;\; (b) $\nt{\nt{A} B}$ \;\;\;\;
(c) $\nt{A + B + C}$ \;\;\;\; (d) $\nt{ABC}$

(e) $\nt{A(B+C)}$ \;\;\;\; (f) $\nt{AB} + \nt{CD}$ \;\;\;\;
(g) $\nt{AB + CD}$ \;\;\;\; (h) $\nt{(A + \nt{B})(\nt{C} + D)}$

\N Aplique os teoremas de DeMorgan a cada expressão:

\vspace{3pt}

(a) $\nt{\nt{A} B(C + \nt{D})}$ \;\;\;\; (b) $\nt{AB (CD + EF)}$

\vspace{3pt}

(c) $\overline{(A + \nt{B} + C + \nt{D})} + \nt{ABC\nt{D}}$ \;\;\;\;
(d) $\nt{ ( \nt{ \nt{A} + B + C + D } )( \nt{ A\nt{B} \; \nt{C}D } ) }$

\vspace{6pt}

(e) $\nt{\nt{AB}(CD + \nt{E}F)(\nt{AB} + \nt{CD})}$

\N Aplique os teoremas de DeMorgan a cada expressão:

\vspace{3pt}

(a) $\nt{ \nt{ (\nt{ABC}) (\nt{EFG}) } + \nt{ (\nt{HIJ}) (\nt{KLM}) } }$ \;\;\;\;
(b) $\nt{(A + \nt{B\nt{C}} + CD)} + \nt{\nt{BC}}$

\vspace{6pt}

(c) $\nt{ \nt{ (\nt{A + B})(\nt{C+D})(\nt{E+F})(\nt{G+H}) }}$

\vspace{6pt}

\setcounter{floyd}{15}

\N Construa a tabela verdade para cada uma das seguintes expressões
booleanas:

(a) $A + B$ \;\;\;\; (b) $AB$ \;\;\;\; (c) $AB + BC$

(d) $(A+B)C$ \;\;\;\; (e) $(A+B)(\nt{B} + C)$

\N Usando técnicas da álgebra booleana, simplifique as expressões
seguintes o máximo possível:

\vspace{3pt}

(a) $A(A+B)$ \;\;\;\; (b) $A(\nt{A} + AB)$ \;\;\;\;
(c) $BC + \nt{B}C$

\vspace{6pt}

(d) $A(A+\nt{A}B)$ \;\;\;\; (e) $A\nt{B}C + \nt{A}BC + \nt{A} \, \nt{B} C$

\N Usando álgebra booleana, simplifique as seguintes expressões:

\vspace{3pt}

(a) $(A + \nt{B})(A+C)$ \;\;\;\;
(b) $\nt{A}B + \nt{A}B\nt{C} + \nt{A}BCD + \nt{A}B\nt{C}\,\nt{D}E$

\vspace{6pt}

(c) $AB + \nt{AB}C + A$ \;\;\;\; (d) $(A + \nt{A})(AB + AB\nt{C})$

\vspace{6pt}

(e) $AB + (\nt{A} + \nt{B})C + AB$

\N Usando álgebra booleana, simplifique cada expressão:

(a) $BD + B(D + E) + \nt{D}(D+F)$ \;\;\;\;
(b) $\nt{A}\,\nt{B}C + \nt{(A+B+\nt{C})} + \nt{A}\,\nt{B}\,\nt{C}D$

(c) $(B+BC)(B + \nt{B}C)(B + D)$ \;\;\;\;
(d) $ABCD + AB(\nt{CD}) + (\nt{AB})CD$

(e) $ABC[AB + \nt{C}(BC + AC)]$

\setcounter{floyd}{20}

\N Converta as expressões seguintes para a forma de soma-de-produtos:

(a) $(A+B)(C + \overline{B})$ \;\;\;\;
(b) $(A + \overline{B}C)C$ \;\;\;\;
(c) $(A+C)(AB+AC)$

\N Converta as expressões seguintes para a forma de soma-de-produtos:

(a) $AB + CD(A\nt{B} + CD)$ \;\;\;\;
(b) $AB(\overline{B} \, \overline{C} + BD)$ \;\;\;\;
(c) $A + B[AC + (B + \nt{C})D]$

\N Defina o domínio de cada expressão do problema 21 e converta-as 
para a forma \textbf{padrão} de soma-de-produtos.

\N Converta cada expressão no problema 22 para a forma \textbf{padrão}
de soma-de-produtos.

\N Determine o valor binário de cada termo na forma padrão de
soma-de-produtos do problema 23.

\N Determine o valor binário de cada termo na forma padrão de
soma-de-produtos do problema 24.

\N Converta cada expressão no Problema 23 da forma padrão de
soma-de-produtos para a forma padrão de produto-de-somas.

\N Converta cada expressão no Problema 24 da forma padrão de
soma-de-produtos para a forma padrão de produto-de-somas.

\N Escreva a tabela verdade para cada uma das seguintes expressões
na forma padrão soma-de-produtos:

\vspace{3pt}

(a) $A \nt{B} C + \nt{A} B \nt{C} + ABC$ \;\;\;\;
(b) $\nt{X}YZ + \nt{X}\,\nt{Y}\,Z + XY\nt{Z} + X\nt{Y}Z + \nt{X}YZ$

\N Escreva a tabela verdade para cada uma das seguintes expressões
na forma padrão soma-de-produtos:

\vspace{3pt}

(a) $\nt{A}B\nt{C}D + \nt{A}BC\nt{D} + A\nt{B}\,\nt{C}D + \nt{A}\,\nt{B}\,\nt{C}\,\nt{D}$

\vspace{6pt}

(b) $WXYZ +WXY\nt{Z} + \nt{W}XYZ + W\nt{X}YZ + WX\nt{Y}Z$

\N Escreva a tabela verdade para cada uma das seguintes expressões
na forma soma-de-produtos (não-padrão):

\vspace{3pt}

(a) $\nt{A}B + AB\nt{C} + \nt{A} \, \nt{C} + A \nt{B} C$ \;\;\;\;

\vspace{6pt}

(b) $\nt{X} + Y\nt{Z} + WZ + X\nt{Y}Z$

%\N Escreva a tabela verdade para cada uma das seguintes expressões
%na forma padrão produto-de-somas:
%
%\vspace{3pt}
%
%(a) $(\nt{A} + \nt{B} + \nt{C})(A + B + C)(A + \nt{B} + C)$
%
%\vspace{6pt}
%
%(b) $(\nt{A} + B + \nt{C} + D)(A + \nt{B} + C + \nt{D})(A + \nt{B} + \nt{C} + D)(\nt{A} + B + C + \nt{D})$
%
%\N Escreva a tabela verdade para cada uma das seguintes expressões
%na forma padrão produto-de-somas:

\setcounter{floyd}{33}

\N Para cada tabela verdade abaixo, determine uma expressão na forma
padrão soma-de-produtos:

(a) \begin{tabular}{c@{}c@{}c|c}
$A$ & $B$ & $C$ & $X$ \\
\hline
 0  &  0  &  0  &  0 \\
 0  &  0  &  1  &  1 \\
 0  &  1  &  0  &  0 \\
 0  &  1  &  1  &  0 \\
 1  &  0  &  0  &  1 \\
 1  &  0  &  1  &  1 \\
 1  &  1  &  0  &  0 \\
 1  &  1  &  1  &  1
\end{tabular}
\hfill
(b) \begin{tabular}{c@{}c@{}c|c}
$A$ & $B$ & $C$ & $X$ \\
\hline
 0  &  0  &  0  &  0 \\
 0  &  0  &  1  &  0 \\
 0  &  1  &  0  &  0 \\
 0  &  1  &  1  &  0 \\
 1  &  0  &  0  &  0 \\
 1  &  0  &  1  &  1 \\
 1  &  1  &  0  &  1 \\
 1  &  1  &  1  &  1
\end{tabular}
\hfill
(c) \begin{tabular}{c@{}c@{}c@{}c|c}
$A$ & $B$ & $C$ & $D$ & $X$ \\
\hline
 0  &  0  &  0  &  0  &  1 \\
 0  &  0  &  0  &  1  &  1 \\
 0  &  0  &  1  &  0  &  0 \\
 0  &  0  &  1  &  1  &  1 \\
 0  &  1  &  0  &  0  &  0 \\
 0  &  1  &  0  &  1  &  1 \\
 0  &  1  &  1  &  0  &  1 \\
 0  &  1  &  1  &  1  &  0 \\
 1  &  0  &  0  &  0  &  0 \\
 1  &  0  &  0  &  1  &  1 \\
 1  &  0  &  1  &  0  &  0 \\
 1  &  0  &  1  &  1  &  0 \\
 1  &  1  &  0  &  0  &  1 \\
 1  &  1  &  0  &  1  &  0 \\
 1  &  1  &  1  &  0  &  0 \\
 1  &  1  &  1  &  1  &  0 \\
\end{tabular}
\hfill
(d) \begin{tabular}{c@{}c@{}c@{}c|c}
$A$ & $B$ & $C$ & $D$ & $X$ \\
\hline
 0  &  0  &  0  &  0  &  0 \\
 0  &  0  &  0  &  1  &  0 \\
 0  &  0  &  1  &  0  &  1 \\
 0  &  0  &  1  &  1  &  0 \\
 0  &  1  &  0  &  0  &  1 \\
 0  &  1  &  0  &  1  &  1 \\
 0  &  1  &  1  &  0  &  0 \\
 0  &  1  &  1  &  1  &  1 \\
 1  &  0  &  0  &  0  &  0 \\
 1  &  0  &  0  &  1  &  0 \\
 1  &  0  &  1  &  0  &  0 \\
 1  &  0  &  1  &  1  &  1 \\
 1  &  1  &  0  &  0  &  1 \\
 1  &  1  &  0  &  1  &  0 \\
 1  &  1  &  1  &  0  &  0 \\
 1  &  1  &  1  &  1  &  1 \\
\end{tabular}

\N Desenhe um mapa de Karnaugh para $3$ variáveis e rotule cada célula de
acordo com o seu valor binário

\N Desenhe um mapa de Karnaugh para $4$ variáveis e rotule cada célula de
acordo com o seu valor binário

\N Escreva o termo-produto padrão para cada célula em um mapa de
Karnaugh de $3$ variáveis.

\N Use o mapa de Karnaugh para encontrar a forma mínima de soma-de-produtos
para cada expressão:

\vspace{3pt}

(a) $\nt{A} \, \nt{B} \, \nt{C} + \nt{A} \, \nt{B} \, C + A \nt{B} C$
\;\;\;\;
(b) $AC(\nt{B} + C)$

\vspace{6pt}

(c) $\nt{A}(BC + B\nt{C}) + A(BC + B\nt{C})$ \;\;\;\;
(d) $\nt{A} \, \nt{B} \, \nt{C} + A \nt{B} \, \nt{C} + \nt{A} B \nt{C} +
     A B \nt{C}$

\N Use o mapa de Karnaugh para simplificar cada expressão para a forma
mínima de soma-de-produtos:

\vspace{3pt}

(a) $\nt{A} \, \nt{B} \, \nt{C} + A \nt{B} C + \nt{A} BC + AB\nt{C}$ \;\;\;\;
(b) $AC[\nt{B} + B(B+\nt{C})]$

\vspace{6pt}

(c) $DE\nt{F} + \nt{D}E\nt{F} + \nt{D}\,\nt{E}\,\nt{F}$

\N Expanda cada expressão para a forma padrão de soma-de-produtos:

\vspace{3pt}

(a) $AB + A\nt{B}C + ABC$ \;\;\;\; (b) $A + BC$

\vspace{6pt}

(c) $A \nt{B} \, \nt{C} D + A C \nt{D} + B \nt{C} D + \nt{A}BC\nt{D}$ \;\;\;\;
(d) $A\nt{B} + A \nt{B} \, \nt{C} D + CD + B \nt{C} D + ABCD$

\N Minimize cada expressão encontrada no Problema 40 usando mapas
de Karnaugh.

\N Use um mapa de Karnaugh para reduzir cada expressão à forma mínima de
soma-de-produtos:

(a) $A + B \nt{C} + CD$

\vspace{6pt}

(b) $\nt{A} \, \nt{B} \, \nt{C} \, \nt{D} + \nt{A} \, \nt{B} \, \nt{C} D +
ABCD + ABC\nt{D}$

\vspace{6pt}

(c) $\nt{A} \, \nt{B} ( \nt{C} \, \nt{D} + \nt{C} D) +
\nt{A} \, \nt{B} ( \nt{C} \, \nt{D} + C \nt{D}) +
A \nt{B} \, \nt{C} D$

\vspace{6pt}

(d) $(\nt{A} \, \nt{B} + A \nt{B})(CD + C\nt{D})$

\vspace{6pt}

(e) $\nt{A} \, \nt{B} + A \nt{B} + CD + C\nt{D}$

\N Reduza a função lógica especificada na tabela verdade abaixo à sua
forma mínima de soma-de-produtos usando um mapa de Karnaugh:

\hspace*{\fill}\begin{tabular}{ccc|c}
\multicolumn{3}{c}{entradas} & saída \\
$A$ & $B$ & $C$ & $X$ \\
\hline
 0  &  0  &  0  &  1 \\
 0  &  0  &  1  &  1 \\
 0  &  1  &  0  &  0 \\
 0  &  1  &  1  &  1 \\
 1  &  0  &  0  &  1 \\
 1  &  0  &  1  &  1 \\
 1  &  1  &  0  &  0 \\
 1  &  1  &  1  &  1
\end{tabular}\hspace*{\fill}

\N Reduza a função lógica especificada na tabela verdade abaixo à sua
forma mínima de soma-de-produtos usando um mapa de Karnaugh:


\hspace*{\fill}\begin{tabular}{c@{\,}c@{\,}c@{\,}c|c}
\multicolumn{4}{c}{entradas} & saída \\
$A$ & $B$ & $C$ & $D$ & $X$ \\
\hline
 0  &  0  &  0  &  0  &  0 \\
 0  &  0  &  0  &  1  &  0 \\
 0  &  0  &  1  &  0  &  1 \\
 0  &  0  &  1  &  1  &  0 \\
 0  &  1  &  0  &  0  &  1 \\
 0  &  1  &  0  &  1  &  1 \\
 0  &  1  &  1  &  0  &  0 \\
 0  &  1  &  1  &  1  &  1 \\
 1  &  0  &  0  &  0  &  0 \\
 1  &  0  &  0  &  1  &  0 \\
 1  &  0  &  1  &  0  &  0 \\
 1  &  0  &  1  &  1  &  1 \\
 1  &  1  &  0  &  0  &  1 \\
 1  &  1  &  0  &  1  &  0 \\
 1  &  1  &  1  &  0  &  0 \\
 1  &  1  &  1  &  1  &  1 \\
\end{tabular}\hspace*{\fill}

\newpage

\textbf{Exercícios adicionais} (não estão no livro-texto)

\questao Obtenha o valor de $X$ nas seguintes expressões lógicas, considerando os seguintes casos: i) $A = 1, B = 1, C = 0, D = 1$; \quad ii) $A = 0, B = 1, C = 0, D = 0$; \\ iii) $A = 1, B = 1, C = 1, D = 1$; \quad iv) $A = 1, B = 0, C = 1, D = 0$
\label{q:explog}

\begin{enumerate}[(a)]
\item $X = A(B \oplus C)$

\item $X = (\overline{A + B}) (C \oplus (A + \overline{D}))$

\item $X = B \overline{C} A + \overline{ ( \overline{C} \oplus D ) }$

\item $X = ( (A + \overline{ \overline{B} \oplus D }) \cdot ( \overline{C} + A) + B ) \cdot \overline{A +  B}$

\item $X = A \oplus B + \overline{C} B + \overline{A}$
\end{enumerate}

\questao Monte as tabelas verdade para cada uma das expressões da questão~\ref{q:explog}.

\questao Use tabelas verdade para demonstrar que as Leis de DeMorgan, listadas abaixo, são válidas:

\begin{enumerate}[(a)] 
\item $\overline{A \cdot B} = \overline{A} + \overline{B}$
\item $\overline{A + B} = \overline{A} \cdot \overline{B}$
\end{enumerate}

\questao Escreva uma expressão para uma função $F(A,B,C,D)$ que é $1$
somente quando:

\begin{enumerate}[(a)]
\item uma única variável é $1$;
\item exatamente duas variáveis são $1$;
\item duas ou três variáveis são $1$;
\item entre duas e quatro variáveis são $1$;
\item até três variáveis são $1$;
\end{enumerate}

\questao Simplifique as seguintes expressões:

\begin{enumerate}[(a)]
\item $XY + XY$
\item $(X + Y) (X + \overline{Y})$
\item $XZ + XY\overline{Z}$
\item $(A + 1)\cdot(B \cdot 0) + D \cdot D + 1$
\item $(A + 1) \cdot B \overline{B} + A + C \cdot C + C \cdot 0 + C$
\end{enumerate}

\questao Simplifique os mapas de Karnaugh abaixo e determine a expressão lógica  como soma de produtos após a simplificação.

\noindent
(a)
\begin{squarecells}{5}
\backslashbox{$x$\kern-1em}{\kern-1em$y z$}
      & $00$ & $01$ & $11$ & $10$ \nl
 $0$  &  1   &  0   &  0   &   1  \nl
 $1$  &  1   &  0   &  1   &   1  \nl
\end{squarecells}
\hspace{.1cm}
(b)
\begin{squarecells}{5}
\backslashbox{$xy$\kern-1em}{\kern-1em$z w$}
       & $00$ & $01$ & $11$ & $10$ \nl
 $00$  &  1   &  1   &  0   &   1  \nl
 $01$  &  1   &  0   &  0   &   1  \nl
 $11$  &  1   &  0   &  0   &   1  \nl
 $10$  &  1   &  1   &  0   &   1  \nl
\end{squarecells}
\hspace{.1cm}
(c)
\begin{squarecells}{5}
\backslashbox{$ab$\kern-1em}{\kern-1em$cd$}
       & $00$ & $01$ & $11$ & $10$ \nl
 $00$  &  1   &  0   &  0   &   1  \nl
 $01$  &  0   &  0   &  1   &   0  \nl
 $11$  &  0   &  1   &  0   &   0  \nl
 $10$  &  1   &  0   &  0   &   1  \nl
\end{squarecells}

\vspace{12pt}

(d)
\begin{squarecells}{5}
\backslashbox{$gh$\kern-1em}{\kern-1em$ij$}
       & $00$ & $01$ & $11$ & $10$ \nl
 $00$  &  0   &  1   &  1   &   0  \nl
 $01$  &  1   &  0   &  0   &   1  \nl
 $11$  &  1   &  0   &  0   &   1  \nl
 $10$  &  0   &  1   &  1   &   0  \nl
\end{squarecells}
\hspace{.1cm}
(e)
\begin{squarecells}{5}
\backslashbox{$gh$\kern-1em}{\kern-1em$ij$}
       & $00$ & $01$ & $11$ & $10$ \nl
 $00$  &  0   &  1   &  1   &   0  \nl
 $01$  &  1   &  1   &  1   &   1  \nl
 $11$  &  1   &  1   &  1   &   1  \nl
 $10$  &  0   &  1   &  1   &   0  \nl
\end{squarecells}

\questao Faça o mapa de Karnaugh para cada uma das expressões na
questão~\ref{q:explog}, efetue as simplificações possíveis e escreva-as
como soma de produtos.

\end{document}

