\documentclass[a4paper,11pt]{article}
\usepackage[utf8]{inputenc}
\usepackage[T1]{fontenc}
\usepackage[portuguese,brazil]{babel}
\usepackage[left=1.5cm,right=1.5cm,top=1.5cm,bottom=1.5cm]{geometry}
\usepackage{amssymb,textcomp}
\usepackage{calc}
\usepackage{tikz}
\usetikzlibrary{arrows,shapes.gates.logic.US,shapes.gates.logic.IEC,calc}
% example: http://tex.stackexchange.com/questions/32839/drawing-circuit-diagrams-with-logic-gates-in-latex

\newcommand{\bit}[1]{\raisebox{-4pt}{\begin{picture}(26,14)\put(0,0){\raisebox{5pt}{\hbox to 26pt{\hfill#1\hfill}}}\put(0,0){\line(1,0){26}}\put(0,0){\line(0,1){14}}\put(26,0){\line(0,1){14}}\put(0,14){\line(1,0){26}}\end{picture}}}
\begin{document}

\setlength{\parindent}{0pt}
\setlength{\parskip}{6pt}

\def\not{\overline}

\begin{center}
\LARGE{Resumão de Circuitos Digitais}
\end{center}

\section{Introdução}

\textbf{Representação de números inteiros}: considere um inteiro representado em uma palavra de dados com $n$ bits \bit{$a_{n-1}$}\bit{$a_{n-2}$}\ldots\bit{$a_{1}$}\bit{$a_0$}. Dependendo do formato adotado, o seu valor será\ldots
\begin{center}
\begin{tabular}{c c}
      formato       & valor \\
\hline
inteiro sem sinal   & $(a_{n-1} a_{n-2} \ldots a_1 a_0)_2$ \\
\hline
sinal-magnitude     & se $a_{n-1} = 0$, $\; +(a_{n-2} \ldots a_1 a_0)_2$ \\
                    & se $a_{n-1} = 1$, $\; -(a_{n-2} \ldots a_1 a_0)_2$ \\
\hline
complemento de dois & se $a_{n-1} = 0$, $\; +(a_{n-2} \ldots a_1 a_0)_2$ \hspace*{5.15ex} \\
                    & se $a_{n-1} = 1$, $\; -[(\overline{a_{n-2}} \ldots \overline{a_1} \; \overline{a_0})_2 + 1]$
\end{tabular}
\end{center}

\textbf{Operações lógicas}: negação (not) $\not{X}$ \hfill conjunção (and) $X \cdot Y$ \hfill disjunção (or) $X+Y$ \\
\phantom{\textbf{Operações lógicas}: }disjunção exclusiva (xor) $X \oplus Y$\\
\phantom{\textbf{Operações lógicas}: }Obs.: $X \oplus Y = \not{X} Y + X \not{Y}$

\textbf{Portas lógicas}: circuitos que executam operações lógicas.

\begin{tabular}{cccc}
    \raisebox{1ex}{
        \begin{tikzpicture}[label distance=2mm]
            \node at (0,0) (input) {$X$};
            \node[not gate US, draw] at (1,0) (gate) {};
            \node at (2.25,0) (output) {$X$};
            \draw (input) -- (gate.input);
            \draw (gate.output) -- (output);
        \end{tikzpicture}
    }
&
    \begin{tikzpicture}
        \node at (0,0.8) (input1) {$X$};
        \node at (0,0.3) (input2) {$Y$};
        \node[and gate US, draw, anchor=input 1, logic gate inputs=nnnn] at (0.85,0.8) (gate) {};
        \node at (3,0.55) (output) {$X \cdot Y$};
        \draw (input1) -- (gate.input 1);
        \draw (input2) -- (gate.input 4);
        \draw (gate.output) -- (output);
    \end{tikzpicture}
&
    \begin{tikzpicture}
        \node at (0,0.8) (input1) {$X$};
        \node at (0,0.3) (input2) {$Y$};
        \node[or gate US, draw, anchor=input 1, logic gate inputs=nnnn] at (0.85,0.8) (gate) {};
        \node at (3.1,0.55) (output) {$X + Y$};
        \draw (input1) -- (gate.input 1);
        \draw (input2) -- (gate.input 4);
        \draw (gate.output) -- (output);
    \end{tikzpicture}
&
    \begin{tikzpicture}
        \node at (0,0.8) (input1) {$X$};
        \node at (0,0.3) (input2) {$Y$};
        \node[xor gate US, draw, scale=1.7, anchor=input 1] at (0.85,0.7) (gate) {};
        \node at (3.3,0.55) (output) {$X \oplus Y$};
        \node at (0.97,0.8) (gtinput1) {};
        \node at (0.97,0.3) (gtinput2) {};
        \draw (input1) -- (gtinput1);
        \draw (input2) -- (gtinput2);
        \draw (gate.output) -- (output);
    \end{tikzpicture}
\\[2ex]
    % blank space
&
    \begin{tikzpicture}
        \node at (0,0.8) (input1) {$X$};
        \node at (0,0.3) (input2) {$Y$};
        \node[nand gate US, draw, anchor=input 1, logic gate inputs=nnnn] at (0.85,0.8) (gate) {};
        \node at (3,0.55) (output) {$\not{X \cdot Y}$};
        \draw (input1) -- (gate.input 1);
        \draw (input2) -- (gate.input 4);
        \draw (gate.output) -- (output);
    \end{tikzpicture}
&
    \begin{tikzpicture}
        \node at (0,0.8) (input1) {$X$};
        \node at (0,0.3) (input2) {$Y$};
        \node[nor gate US, draw, anchor=input 1, logic gate inputs=nnnn] at (0.85,0.8) (gate) {};
        \node at (3.1,0.55) (output) {$\not{X + Y}$};
        \draw (input1) -- (gate.input 1);
        \draw (input2) -- (gate.input 4);
        \draw (gate.output) -- (output);
    \end{tikzpicture}
&
    \begin{tikzpicture}
        \node at (0,0.8) (input1) {$X$};
        \node at (0,0.3) (input2) {$Y$};
        \node[xnor gate US, draw, scale=1.7, anchor=input 1] at (0.85,0.7) (gate) {};
        \node at (3.4,0.55) (output) {$\not{X \oplus Y}$};
        \node at (0.97,0.8) (gtinput1) {};
        \node at (0.97,0.3) (gtinput2) {};
        \draw (input1) -- (gtinput1);
        \draw (input2) -- (gtinput2);
        \draw (gate.output) -- (output);
    \end{tikzpicture}
\end{tabular}

\section{Síntese de circuitos digitais combinacionais}

0. Entenda o problema e obtenha a tabela verdade para cada saída.\\
1. Obtenha e simplifique a expressão lógica para cada saída. Para até $4$ variáveis (ou até $6$ para os muito atentos), é possível simplificar manualmente usando mapas de Karnaugh.\\
2. Desenhe o diagrama do circuito.\\
3. Analise o circuito e verifique as saídas.\\
4. Simulação e montagem (não dá para fazer na prova).

\section{Blocos lógicos básicos}

\begin{minipage}{0.6\textwidth}
\textbf{Meio-somador (half-adder)}
\begin{itemize}
\setlength{\itemsep}{-2pt}
    \item Entradas: $a_0$, $b_0$, dois bits a serem adicionados.
    \item Saídas: $s_0$, a soma aritmética das entradas;\\
    \phantom{Saídas: }$c_{out}$, vai-um.
\end{itemize}
\end{minipage}
\begin{minipage}{0.4\textwidth}
\begin{center}
\begin{picture}(80,90)
\put(27,75){\line(0,1){15}}   \put(51,75){\line(0,1){15}}
\put(23,64){$a_0$}            \put(47,64){$b_0$}
\put(0,46){\line(1,0){15}}  \put(19,38){\rotatebox{90}{\small{$c_{out}$}}}
\put(32,41){\large{HA}}
%\put(50,44){\line(0,1){2}}
%\put(49,45){\line(1,0){2}}
\put(35,20){$s_0$}
\put(40,15){\line(0,-1){15}}
\put(15,15){\framebox(50,60){}}
\end{picture}
\end{center}
\end{minipage}

\vspace{12pt}

\begin{minipage}{0.6\textwidth}
\textbf{Somador completo (full-adder)}
\begin{itemize}
\setlength{\itemsep}{-2pt}
    \item Entradas: $a_i$, $b_i$, $c_{in}$, três bits a serem somados.
    \item Saídas: $s_i$, soma aritmética das entradas;\\
          \phantom{Saídas: }$c_{out}$, vai-um.
\end{itemize}
\end{minipage}
\begin{minipage}{0.4\textwidth}
\begin{center}
\begin{picture}(80,90)
\put(27,75){\line(0,1){15}}   \put(51,75){\line(0,1){15}}
\put(23,64){$a_0$}            \put(47,64){$b_0$}
\put(0,46){\line(1,0){15}}  \put(19,38){\rotatebox{90}{\small{$c_{out}$}}} \put(55,40){\rotatebox{90}{\small{$c_{in}$}}} \put(65,46){\line(1,0){15}}
\put(32,41){\large{FA}}
%\put(50,44){\line(0,1){2}}
%\put(49,45){\line(1,0){2}}
\put(35,20){$s_0$}
\put(40,15){\line(0,-1){15}}
\put(15,15){\framebox(50,60){}}
\end{picture}
\end{center}
\end{minipage}

\newpage

\textbf{Somador completo de $n$ bits}
\begin{itemize}
\setlength{\itemsep}{-2pt}
    \item Entradas: $a_{n-1}$, $a_{n-2}$, \ldots, $a_1$, $a_0$, bits que representam um inteiro $A$;\\
    \phantom{Entradas: }$b_{n-1}$, $b_{n-2}$, \ldots, $b_1$, $b_0$, bits que representam um inteiro $B$;\\
    \phantom{Entradas: }$c_{in}$, vem-um.
    \item Saídas: $s_{n-1}$, $s_{n-2}$, \ldots, $s_1$, $s_0$, representando o resultado da soma aritmética $A+B+c_{in}$;\\
    \phantom{Saídas: }$c_{out}$, vai-um.
\end{itemize}
Os inteiros $A$ e $B$ ambos devem estar no formato inteiro sem sinal ou no formato complemento de dois. A saída estará no mesmo formato da entrada.

\textbf{Decodificador $k$ para $2^k$} -- Entradas: $a_{k-1}$, $a_{k-2}$, \ldots, $a_1$, $a_0$, as entradas de endereço. Saídas: $X_0, X_1$, \ldots, $X_{2^k-1}$, onde cada combinação de bits da entrada ativa apenas uma saída $X_{j}$ tal que $j = (a_{k-1} a_{k-2} \ldots a_1 a_0)_2$.

\textbf{Multiplexador (mux) $2^k$ para $1$} -- Entradas: $I_0$, $I_1$, \ldots, $I_{2^k-1}$, as entradas de dados; $s_{k-1}$, $s_{k-2}$, \ldots, $s_1$, $s_0$, as entradas de seleção. Saída: $X$, onde $X$ tem o valor da entrada $I_j$ tal que $j = (s_{k-1} s_{k-2} \ldots s_1 s_0)_2$. O mux funciona como uma chave seletora.

\section{Circuitos digitais sequenciais}

\newlength{\lengthOfDiagdown}
\settowidth{\lengthOfDiagdown}{$\diagdown$}
\def\X{$\diagdown$\hspace{-\lengthOfDiagdown}$\diagup$}

\def\CK{%
\begin{picture}(10,10)%
\thicklines%
\put(0,10){\line(1,0){5}}%
\put(5,10){\line(0,-1){10}}%
\put(5,0){\line(1,0){5}}%
\put(5,10){\vector(0,-1){9}}%
\end{picture}%
}

Notação: $Q_i$ = estado atual da variável $Q$; \hspace{3ex} $Q_{i-1}$ = estado anterior da variável $Q$; \hspace{3ex} \X{} = \emph{don't care}

\textbf{Latches}: circuitos digitais sequenciais básicos, que guardam informação sobre o estado anterior. Cada um dos latches abaixo possui uma saída $Q$, e uma saída com estado inverso $\not{Q}$.

\vspace{12pt}

\begin{minipage}{0.32\textwidth}
\begin{center}
Latch R-S (Reset-Set)\\[6pt]

\begin{picture}(100,63)
\put(10,52){\line(1,0){15}} \put(30,48){$R$} \put(60,48){$Q$} \put(75,52){\line(1,0){15}}
\put(10,16){\line(1,0){15}} \put(30,12){$S$} \put(60,12){$\overline{Q}$} \put(75,16){\line(1,0){15}} 
\put(25,3){\framebox(50,60){}}
\end{picture}\\[6pt]

\begin{tabular}{cc||cl}
$R$ & $S$ & $Q_i$ & \\
\cline{1-4}
 0  &  0  & $Q_{i-1}$ & \footnotesize{(mantém)} \\
 0  &  1  &   1   & \footnotesize{(set)} \\
 1  &  0  &   0   & \footnotesize{(reset)} \\
 1  &  1  & \multicolumn{2}{c}{entrada proibida}
\end{tabular}
\end{center}
\end{minipage}%
%
\vline%
%
\begin{minipage}{0.36\textwidth}
\begin{center}
Latch R-S c/ Enable\\[6pt]

\begin{picture}(100,63)
\put(10,52){\line(1,0){15}} \put(30,48){$R$} \put(60,48){$Q$} \put(75,52){\line(1,0){15}}
\put(10,34){\line(1,0){15}} \put(30,30){$En$}
\put(10,16){\line(1,0){15}} \put(30,12){$S$} \put(60,12){$\overline{Q}$} \put(75,16){\line(1,0){15}} 
\put(25,3){\framebox(50,60){}}
\end{picture}\\[6ex]

    \begin{minipage}{0.8\textwidth}
     Quando $En = 0$, as saídas não se alteram. Quando $En = 1$, funciona
     como o latch R-S comum.
    \end{minipage}

\end{center}
\end{minipage}%
%
\vline%
%
\begin{minipage}{0.32\textwidth}
\begin{center}
Latch D (Data)\\[6pt]

\begin{picture}(100,63)
\put(10,52){\line(1,0){15}} \put(30,48){$D$} \put(60,48){$Q$} \put(75,52){\line(1,0){15}}
\put(10,16){\line(1,0){15}} \put(30,12){$En$} \put(60,12){$\overline{Q}$} \put(75,16){\line(1,0){15}} 
\put(25,3){\framebox(50,60){}}
\end{picture}\\[6pt]

\begin{tabular}{cc||cl}
$En$ & $D$ & $Q_i$ & \\
\cline{1-4}
 0  &  0  & $Q_{i-1}$ & \footnotesize{(mantém)} \\
 0  &  1  & $Q_{i-1}$ & \footnotesize{(mantém)} \\
 1  &  0  &   0   & \footnotesize{(reset)} \\
 1  &  1  &   1   & \footnotesize{(set)} \\
\end{tabular}
\end{center}
\end{minipage}

\vspace{12pt}

\def\dowCK{$1\!\rightarrow\!0$}

\textbf{Flip-flops}: latches sensíveis a uma das bordas do \emph{clock}. O símbolo \CK{} denota sensibilidade à borda de descida (transição $1 \rightarrow 0$).

\vspace{12pt}

\begin{minipage}{0.32\textwidth}
\begin{center}
Flip-flop R-S\\[6pt]

\begin{picture}(120,63)(-10,0)
\put(10,52){\line(1,0){15}} \put(30,48){$R$} \put(60,48){$Q$} \put(75,52){\line(1,0){15}}
\put(-10,30){$CK$}\put(10,34){\line(1,0){15}} \put(30,30){\CK}
\put(10,16){\line(1,0){15}} \put(30,12){$S$} \put(60,12){$\overline{Q}$} \put(75,16){\line(1,0){15}} 
\put(25,3){\framebox(50,60){}}
\end{picture}\\[6pt]

\begin{tabular}{c@{\ \ }c@{\ \ }c||cl}
$R$ & $S$ &  $CK$  & $Q_i$ & \\
\cline{1-4}
 0  &  0  &   \X   & $Q_{i-1}$ & \footnotesize{(mantém)} \\
 0  &  1  & \dowCK &   1   & \footnotesize{(set)} \\
 1  &  0  & \dowCK &   0   & \footnotesize{(reset)} \\
 1  &  1  & \dowCK & \multicolumn{2}{c}{entrada proibida}
\end{tabular}
\end{center}
\end{minipage}%
%
\vline%
%
\begin{minipage}{0.32\textwidth}
\begin{center}
Flip-flop D\\[6pt]

\begin{picture}(120,63)(-10,0)
\put(10,52){\line(1,0){15}} \put(30,48){$D$} \put(60,48){$Q$} \put(75,52){\line(1,0){15}}
\put(-10,12){$CK$} \put(10,16){\line(1,0){15}} \put(30,12){\CK} \put(60,12){$\overline{Q}$} \put(75,16){\line(1,0){15}} 
\put(25,3){\framebox(50,60){}}
\end{picture}\\[6pt]

\vspace{5.8ex}

\begin{tabular}{cc||cl}
$D$ & $CK$ & $Q_i$ & \\
\cline{1-4}
 0  & \dowCK  &   0   & \footnotesize{(reset)} \\
 1  & \dowCK  &   1   & \footnotesize{(set)} \\
\end{tabular}
\end{center}
\end{minipage}%
%
\vline%
%
\begin{minipage}{0.32\textwidth}
\begin{center}
Flip-flop J-K (Jump-Kill)\\[6pt]

\begin{picture}(120,63)(-10,0)
\put(10,52){\line(1,0){15}} \put(30,48){$J$} \put(60,48){$Q$} \put(75,52){\line(1,0){15}}
\put(-10,30){$CK$}\put(10,34){\line(1,0){15}} \put(30,30){\CK}
\put(10,16){\line(1,0){15}} \put(30,12){$K$} \put(60,12){$\overline{Q}$} \put(75,16){\line(1,0){15}} 
\put(25,3){\framebox(50,60){}}
\end{picture}\\[6pt]

\begin{tabular}{c@{\ \ }c@{\ \ }c||cl}
$J$ & $K$ &  $CK$  & $Q_i$     & \\
\cline{1-4}
 0  &  0  &   \X   & $Q_{i-1}$     & \footnotesize{(mantém)} \\
 0  &  1  & \dowCK &   0       & \footnotesize{(kill)} \\
 1  &  0  & \dowCK &   1       & \footnotesize{(jump)} \\
 1  &  1  & \dowCK & $\not{Q_{i-1}}$ & \footnotesize{(inverte)} \\
\end{tabular}
\end{center}
\end{minipage}

\newpage

\def\pageW{510}
\def\pageH{756}

\def\cross{\hspace{-2pt}\line(1,0){4}\hspace{-2pt}\raisebox{2pt}{\line(0,-1){4}}}

\begin{picture}(\pageW,\pageH)
{\color{gray}
\thinlines
\multiput(0,0)(4,0){130}{\line(0,1){\pageH}}
\multiput(0,0)(0,4){190}{\line(1,0){\pageW}}
}
\put(0,0){\scalebox{4}{Somador $n$ bits}}

\newlength{\tmpX}
\newlength{\tmpY}
\newlength{\leadLabelDistance}
\setlength{\leadLabelDistance}{3pt}

% \southLead{x}{y}{length}{label}
\newcommand{\southLead}[4]{
\settowidth{\tmpX}{#4}
\settoheight{\tmpY}{#4}
\addtolength{\tmpY}{\leadLabelDistance}

\put(#1,#2){%
\scalebox{4}{%
\raisebox{15pt}{\vector(0,-1){15}}% draw lead
\hspace{-0.5\tmpX}% center
\raisebox{-\tmpY}{#4}%
\hspace{-0.5\tmpX}% center
}% end scalebox
}% end put
} %end newcommand

\southlead{50}{50}{$a_0$}

\thicklines
\multiput(0,0)(40,0){13}{\line(0,1){\pageH}}
\multiput(0,0)(0,40){19}{\line(1,0){\pageW}}
\end{picture}

\end{document}
