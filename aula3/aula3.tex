% This tex file is available under a
% Creative Commons Attribution-Share Alike license (CC BY-SA 2.0).
% http://creativecommons.org/licenses/by-sa/2.0/
% Copyright © 2013 Rodrigo Hausen
\documentclass{beamer}
\usepackage[utf8]{inputenc}
\usepackage{lmodern}
\usepackage[T1]{fontenc}
\usepackage[portuguese,brazil]{babel}
\usepackage{url}
\usepackage{listings}
\usepackage{color}
\usepackage{textcomp}
\usepackage{pdfpages}
\usepackage{fancyvrb}
\usepackage{enumerate}
\usepackage{icomma} % para vírgula decimal / decimal comma
\definecolor{listinggray}{gray}{0.9}
\definecolor{lbcolor}{rgb}{0.9,0.9,0.9}
\definecolor{mediumgray}{rgb}{0.6,0.6,0.6}
\lstset{
    backgroundcolor=\color{lbcolor},
    tabsize=4,
    rulecolor=,
    basicstyle=\scriptsize,
    upquote=true,
    aboveskip={1.5\baselineskip},
    columns=fixed,
    showstringspaces=false,
    extendedchars=true,
    breaklines=true,
    prebreak = \raisebox{0ex}[0ex][0ex]{\ensuremath{\hookleftarrow}},
    frame=single,
    showtabs=false,
    showspaces=false,
    showstringspaces=false,
    identifierstyle=\ttfamily,
    keywordstyle=\color[rgb]{0,0,1},
    commentstyle=\color[rgb]{0.133,0.545,0.133},
    stringstyle=\color[rgb]{0.627,0.126,0.941},
}

\definecolor{pinegreen}{RGB}{0,139,114}
\newcommand{\comment}[1]{{\color{structure.fg!70!white}\footnotesize #1}}

\newcommand{\WD}[1]{\fbox{#1}\hspace{-0.5pt}}

\def\A{\texttt{A}}
\def\B{\texttt{B}}
\def\C{\texttt{C}}
\def\D{\texttt{D}}
\def\E{\texttt{E}}
\def\F{\texttt{F}}

\usetheme{Boadilla}
%\usetheme{umbc2}
\usefonttheme{structuresmallcapsserif}
\usecolortheme{seahorse}

\title{Aula 3: Aritmética, Representação de dados}
\subtitle{Circuitos Digitais}
\author{Rodrigo Hausen}
\institute{CMCC -- UFABC} 
\date{28 e 30 de janeiro de 2013}

\begin{document}

\begin{frame}
\maketitle

\vspace{-1cm}

\begin{center}
\url{http://compscinet.org/circuitos}
\end{center}

\end{frame}

%%%%%%%%%%%%%%%%%%%%%%%%%%%%%%%%%%%%%%%%%%%%

\begin{frame}[fragile]
\frametitle{Multiplicação binária}

\begin{itemize}
\item Algoritmo da multiplicação: mesma idéia usada na base decimal.
\begin{Verbatim}[commandchars=\\\{\},codes={\catcode`$=3\catcode`^=7}]
        11011
   x \underline{     101}
        11011
       00000
   + \underline{ 11011  }
     10000111
\end{Verbatim}
\pause
\item Note que a tabuada da multiplicação na base $2$ é muito mais fácil.
\begin{center}
\begin{tabular}{c|cc}
$\times$ & 0 & 1 \\
\hline
       0 & 0 & 0 \\
       1 & 0 & 1
\end{tabular}
\end{center}
\pause
\item Se $A$ tem $n$ algarismos e $B$ tem $m$ algarismos, então o produto $A \times B$ terá, no máximo, \pause $n+m$ algarismos.
\end{itemize}

\end{frame}

\begin{frame}[fragile]
\frametitle{Multiplicação binária}

\textbf{Para casa}: escrever o algoritmo de multiplicação binária para números naturais.

\newcommand{\onlytwo}[1]{\only<2>{#1}}
\newcommand{\onlyth}[1]{\only<3-4>{#1}}
\newcommand{\onlyfo}[1]{\only<4>{#1}}
\newcommand{\onlyfi}[1]{\only<5>{#1}}

\begin{itemize}
\item Note que não é necessário armazenar todas as parcelas da soma ao mesmo tempo.
\begin{Verbatim}[commandchars=\\\{\},codes={\catcode`$=3\catcode`^=7}]
        11011
   x \underline{     101}
     \onlytwo{   11011}\onlyth{  011011}\onlyfi{10000111}
   \onlytwo{+   00000{\color{green}0} <- desloca 1}\onlyfo{+  11011{\color{green}00}<- desloca 2}
\end{Verbatim}
\end{itemize}
\end{frame}

%%%%%%%%%%%%%
% DIVISAO

   %%%%%%%%%%

\begin{frame}[fragile]
\frametitle{Divisão binária}

\begin{itemize}
\item Algoritmo da divisão longa: de novo, emprestamos a idéia da base decimal.
\pause
\item Novamente, a tabuada binária facilita as contas. Os algarismos só podem ser $1$ ou $0$.
\end{itemize}

\def\hs{\hspace{-1mm}}

\begin{Verbatim}[commandchars=\\\{\},codes={\catcode`$=3\catcode`^=7}]
      10000111  |\hs\underline{  101  }


\end{Verbatim}

\end{frame}

   %%%%%%%%%%

\begin{frame}[fragile]
\frametitle{Divisão binária}

\begin{itemize}
\item Algoritmo da divisão longa: de novo, emprestamos a idéia da base decimal.

\item Novamente, a tabuada binária facilita as contas. Os algarismos só podem ser $1$ ou $0$.
\end{itemize}

\def\hs{\hspace{-1mm}}

\begin{Verbatim}[commandchars=\\\{\},codes={\catcode`$=3\catcode`^=7}]
      10000111  |\hs\underline{  101  }
    - \underline{101}
       -1
\end{Verbatim}

\end{frame}

   %%%%%%%%%%

\begin{frame}[fragile]
\frametitle{Divisão binária}

\begin{itemize}
\item Algoritmo da divisão longa: de novo, emprestamos a idéia da base decimal.

\item Novamente, a tabuada binária facilita as contas. Os algarismos só podem ser $1$ ou $0$.
\end{itemize}

\def\hs{\hspace{-1mm}}

\begin{Verbatim}[commandchars=\\\{\},codes={\catcode`$=3\catcode`^=7}]
      10000111  |\hs\underline{  101  }
    - \underline{101}        0
       -1
\end{Verbatim}

\end{frame}

   %%%%%%%%%%

\begin{frame}[fragile]
\frametitle{Divisão binária}

\begin{itemize}
\item Algoritmo da divisão longa: de novo, emprestamos a idéia da base decimal.

\item Novamente, a tabuada binária facilita as contas. Os algarismos só podem ser $1$ ou $0$.
\end{itemize}

\def\hs{\hspace{-1mm}}

\begin{Verbatim}[commandchars=\\\{\},codes={\catcode`$=3\catcode`^=7}]
      10000111  |\hs\underline{  101  }
     - \underline{101}       0
        
\end{Verbatim}

\end{frame}


   %%%%%%%%%%

\begin{frame}[fragile]
\frametitle{Divisão binária}

\begin{itemize}
\item Algoritmo da divisão longa: de novo, emprestamos a idéia da base decimal.

\item Novamente, a tabuada binária facilita as contas. Os algarismos só podem ser $1$ ou $0$.
\end{itemize}

\def\hs{\hspace{-1mm}}

\begin{Verbatim}[commandchars=\\\{\},codes={\catcode`$=3\catcode`^=7}]
      10000111  |\hs\underline{  101  }
     - \underline{101}       0
        11
\end{Verbatim}

\end{frame}

   %%%%%%%%%%

\begin{frame}[fragile]
\frametitle{Divisão binária}

\begin{itemize}
\item Algoritmo da divisão longa: de novo, emprestamos a idéia da base decimal.

\item Novamente, a tabuada binária facilita as contas. Os algarismos só podem ser $1$ ou $0$.
\end{itemize}

\def\hs{\hspace{-1mm}}

\begin{Verbatim}[commandchars=\\\{\},codes={\catcode`$=3\catcode`^=7}]
      10000111  |\hs\underline{  101  }
     - \underline{101}       01
        11
\end{Verbatim}

\end{frame}

   %%%%%%%%%%

\begin{frame}[fragile]
\frametitle{Divisão binária}

\begin{itemize}
\item Algoritmo da divisão longa: de novo, emprestamos a idéia da base decimal.

\item Novamente, a tabuada binária facilita as contas. Os algarismos só podem ser $1$ ou $0$.
\end{itemize}

\def\hs{\hspace{-1mm}}

\begin{Verbatim}[commandchars=\\\{\},codes={\catcode`$=3\catcode`^=7}]
        110111  |\hs\underline{  101  }
                 01
            
\end{Verbatim}

\end{frame}

   %%%%%%%%%%

\begin{frame}[fragile]
\frametitle{Divisão binária}

\begin{itemize}
\item Algoritmo da divisão longa: de novo, emprestamos a idéia da base decimal.

\item Novamente, a tabuada binária facilita as contas. Os algarismos só podem ser $1$ ou $0$.
\end{itemize}

\def\hs{\hspace{-1mm}}

\begin{Verbatim}[commandchars=\\\{\},codes={\catcode`$=3\catcode`^=7}]
        110111  |\hs\underline{  101  }
      - \underline{101}      01
          1  
\end{Verbatim}

\end{frame}

   %%%%%%%%%%

\begin{frame}[fragile]
\frametitle{Divisão binária}

\begin{itemize}
\item Algoritmo da divisão longa: de novo, emprestamos a idéia da base decimal.

\item Novamente, a tabuada binária facilita as contas. Os algarismos só podem ser $1$ ou $0$.
\end{itemize}

\def\hs{\hspace{-1mm}}

\begin{Verbatim}[commandchars=\\\{\},codes={\catcode`$=3\catcode`^=7}]
        110111  |\hs\underline{  101  }
      - \underline{101}      011
          1  
\end{Verbatim}

\end{frame}

   %%%%%%%%%%

\begin{frame}[fragile]
\frametitle{Divisão binária}

\begin{itemize}
\item Algoritmo da divisão longa: de novo, emprestamos a idéia da base decimal.

\item Novamente, a tabuada binária facilita as contas. Os algarismos só podem ser $1$ ou $0$.
\end{itemize}

\def\hs{\hspace{-1mm}}

\begin{Verbatim}[commandchars=\\\{\},codes={\catcode`$=3\catcode`^=7}]
          1111  |\hs\underline{  101  }
       - \underline{101}     011
         -10  
\end{Verbatim}

\end{frame}

   %%%%%%%%%%

\begin{frame}[fragile]
\frametitle{Divisão binária}

\begin{itemize}
\item Algoritmo da divisão longa: de novo, emprestamos a idéia da base decimal.

\item Novamente, a tabuada binária facilita as contas. Os algarismos só podem ser $1$ ou $0$.
\end{itemize}

\def\hs{\hspace{-1mm}}

\begin{Verbatim}[commandchars=\\\{\},codes={\catcode`$=3\catcode`^=7}]
          1111  |\hs\underline{  101  }
       - \underline{101}     0110
         -10  
\end{Verbatim}

\end{frame}

   %%%%%%%%%%

\begin{frame}[fragile]
\frametitle{Divisão binária}

\begin{itemize}
\item Algoritmo da divisão longa: de novo, emprestamos a idéia da base decimal.

\item Novamente, a tabuada binária facilita as contas. Os algarismos só podem ser $1$ ou $0$.
\end{itemize}

\def\hs{\hspace{-1mm}}

\begin{Verbatim}[commandchars=\\\{\},codes={\catcode`$=3\catcode`^=7}]
          1111  |\hs\underline{  101  }
        - \underline{101}    01101
           10  
\end{Verbatim}

\end{frame}

   %%%%%%%%%%

\begin{frame}[fragile]
\frametitle{Divisão binária}

\begin{itemize}
\item Algoritmo da divisão longa: de novo, emprestamos a idéia da base decimal.

\item Novamente, a tabuada binária facilita as contas. Os algarismos só podem ser $1$ ou $0$.
\end{itemize}

\def\hs{\hspace{-1mm}}

\begin{Verbatim}[commandchars=\\\{\},codes={\catcode`$=3\catcode`^=7}]
           101  |\hs\underline{  101  }
         - \underline{101}   011011
             0
\end{Verbatim}

\end{frame}

   %%%%%%%%%%

\begin{frame}[fragile]
\frametitle{Divisão binária}

\begin{itemize}
\item Algoritmo da divisão longa: de novo, emprestamos a idéia da base decimal.

\item Novamente, a tabuada binária facilita as contas. Os algarismos só podem ser $1$ ou $0$.
\end{itemize}

\def\hs{\hspace{-1mm}}

\begin{Verbatim}[commandchars=\\\{\},codes={\catcode`$=3\catcode`^=7}]
           101  |\hs\underline{  101  }
         - \underline{101}   011011 \textsf{= quociente}
             0 \textsf{= resto}
\end{Verbatim}

\end{frame}

% /DIVISAO
%%%%%%%%%%%%%

%%%%%%%%%%%%%
% DIVISAO ALGORITMO

\begin{frame}[fragile]
\frametitle{Divisão binária}

\textbf{Para casa}: escrever, em pseudocódigo, o algoritmo da divisão binária para números naturais.

\begin{itemize}
\item Calcular $A \div B$, onde $A = a_{n-1} \ldots a_0$, $B = b_{m-1} \ldots b_0$ e $m \le n$
\item Note que as subtrações ``da esquerda para a direita'', são, na verdade, subtrações do dividendo pelo divisor multiplicado por $2^i$, para $i = n-m \ldots 0$
\end{itemize}

\def\hs{\hspace{-1mm}}

\begin{Verbatim}[commandchars=\\\{\},codes={\catcode`$=3\catcode`^=7}]
      10000111  |\hs\underline{  101  }
    - \underline{101{\color{green}00000}}
       -100111
\end{Verbatim}

\vspace{44pt}

\end{frame}

     %%%%%%%%%%%%%

\begin{frame}[fragile]
\frametitle{Divisão binária}

\textbf{Para casa}: escrever, em pseudocódigo, o algoritmo da divisão binária para números naturais.

\begin{itemize}
\item Calcular $A \div B$, onde $A = a_{n-1} \ldots a_0$, $B = b_{m-1} \ldots b_0$ e $m \le n$
\item Note que as subtrações ``da esquerda para a direita'', são, na verdade, subtrações do dividendo pelo divisor multiplicado por $2^i$, para $i = n-m \ldots 0$
\end{itemize}

\def\hs{\hspace{-1mm}}

\begin{Verbatim}[commandchars=\\\{\},codes={\catcode`$=3\catcode`^=7}]
      10000111  |\hs\underline{  101  }
    - \underline{101{\color{green}00000}}   0
       -100111
\end{Verbatim}

\vspace{44pt}

\end{frame}

     %%%%%%%%%%%%%

\begin{frame}[fragile]
\frametitle{Divisão binária}

\textbf{Para casa}: escrever, em pseudocódigo, o algoritmo da divisão binária para números naturais.

\begin{itemize}
\item Calcular $A \div B$, onde $A = a_{n-1} \ldots a_0$, $B = b_{m-1} \ldots b_0$ e $m \le n$
\item Note que as subtrações ``da esquerda para a direita'', são, na verdade, subtrações do dividendo pelo divisor multiplicado por $2^i$, para $i = n-m \ldots 0$
\end{itemize}

\def\hs{\hspace{-1mm}}

\begin{Verbatim}[commandchars=\\\{\},codes={\catcode`$=3\catcode`^=7}]
      10000111  |\hs\underline{  101  }
     - \underline{101{\color{green}0000}}   0
        110111 
\end{Verbatim}

\vspace{44pt}

\end{frame}

     %%%%%%%%%%%%%

\begin{frame}[fragile]
\frametitle{Divisão binária}

\textbf{Para casa}: escrever, em pseudocódigo, o algoritmo da divisão binária para números naturais.

\begin{itemize}
\item Calcular $A \div B$, onde $A = a_{n-1} \ldots a_0$, $B = b_{m-1} \ldots b_0$ e $m \le n$
\item Note que as subtrações ``da esquerda para a direita'', são, na verdade, subtrações do dividendo pelo divisor multiplicado por $2^i$, para $i = n-m \ldots 0$
\end{itemize}

\def\hs{\hspace{-1mm}}

\begin{Verbatim}[commandchars=\\\{\},codes={\catcode`$=3\catcode`^=7}]
      10000111  |\hs\underline{  101  }
     - \underline{101{\color{green}0000}}   01
        110111 
\end{Verbatim}

\vspace{44pt}

\end{frame}

     %%%%%%%%%%%%%

\begin{frame}[fragile]
\frametitle{Divisão binária}

\textbf{Para casa}: escrever, em pseudocódigo, o algoritmo da divisão binária para números naturais.

\begin{itemize}
\item Calcular $A \div B$, onde $A = a_{n-1} \ldots a_0$, $B = b_{m-1} \ldots b_0$ e $m \le n$
\item Note que as subtrações ``da esquerda para a direita'', são, na verdade, subtrações do dividendo pelo divisor multiplicado por $2^i$, para $i = n-m \ldots 0$
\end{itemize}

\def\hs{\hspace{-1mm}}

\begin{Verbatim}[commandchars=\\\{\},codes={\catcode`$=3\catcode`^=7}]
        110111  |\hs\underline{  101  }
      - \underline{101{\color{green}000}}   01
         01111 
\end{Verbatim}

\vspace{44pt}

\end{frame}

     %%%%%%%%%%%%%

\begin{frame}[fragile]
\frametitle{Divisão binária}

\textbf{Para casa}: escrever, em pseudocódigo, o algoritmo da divisão binária para números naturais.

\begin{itemize}
\item Calcular $A \div B$, onde $A = a_{n-1} \ldots a_0$, $B = b_{m-1} \ldots b_0$ e $m \le n$
\item Note que as subtrações ``da esquerda para a direita'', são, na verdade, subtrações do dividendo pelo divisor multiplicado por $2^i$, para $i = n-m \ldots 0$
\end{itemize}

\def\hs{\hspace{-1mm}}

\begin{Verbatim}[commandchars=\\\{\},codes={\catcode`$=3\catcode`^=7}]
        110111  |\hs\underline{  101  }
      - \underline{101{\color{green}000}}   011
         01111 
\end{Verbatim}

\vspace{44pt}

\end{frame}

     %%%%%%%%%%%%%

\begin{frame}[fragile]
\frametitle{Divisão binária}

\textbf{Para casa}: escrever, em pseudocódigo, o algoritmo da divisão binária para números naturais.

\begin{itemize}
\item Calcular $A \div B$, onde $A = a_{n-1} \ldots a_0$, $B = b_{m-1} \ldots b_0$ e $m \le n$
\item Note que as subtrações ``da esquerda para a direita'', são, na verdade, subtrações do dividendo pelo divisor multiplicado por $2^i$, para $i = n-m \ldots 0$
\end{itemize}

\def\hs{\hspace{-1mm}}

\begin{Verbatim}[commandchars=\\\{\},codes={\catcode`$=3\catcode`^=7}]
         01111  |\hs\underline{  101  }
       - \underline{101{\color{green}00}}   0110\textsf{\ldots}
         -\textsf{\ldots}
\end{Verbatim}

\vspace{44pt}

\end{frame}

     %%%%%%%%%%%%%

\begin{frame}[fragile]
\frametitle{Divisão binária}

\textbf{Para casa}: escrever, em pseudocódigo, o algoritmo da divisão binária para números naturais.

\begin{itemize}
\item Calcular $A \div B$, onde $A = a_{n-1} \ldots a_0$, $B = b_{m-1} \ldots b_0$ e $m \le n$
\item Note que as subtrações ``da esquerda para a direita'', são, na verdade, subtrações do dividendo pelo divisor multiplicado por $2^i$, para $i = n-m \ldots 0$
\end{itemize}

\def\hs{\hspace{-1mm}}

\begin{Verbatim}[commandchars=\\\{\},codes={\catcode`$=3\catcode`^=7}]
         01111  |\hs\underline{  101  }
       - \underline{101{\color{green}00}}   0110\textsf{\ldots}
         -\textsf{\ldots}
\end{Verbatim}

\begin{itemize}
\item Se a diferença é positiva, ela passa a ser o próximo dividendo.
\item Pare quando $i = 0$.
\end{itemize}

\end{frame}


% /DIVISAO ALGORITMO
%%%%%%%%%%%%%

\begin{frame}[fragile]
\frametitle{Divisão binária: divisão não-inteira}

\begin{itemize}
\item Note que se $A$ é múltiplo de $B$, o resultado da última subtração será $0$
\pause
\item E se $A$ não for múltiplo de $B$?
\pause
\item Podemos continuar a divisão, adicionando a vírgula (ponto, em inglês).
\pause
\item para cada algarismo adicionado depois da vírgula, multiplique o dividendo por $2$ (ou seja, adicione um $0$ à direita)
\pause
\item Pare quando o resultado tiver $k$ algarismos depois da vírgula.
\end{itemize}

Ex.: $110 \div 101$\\[12pt]

\pause

\textbf{Para casa}: escrever esse algoritmo.
\end{frame}

\begin{frame}
\frametitle{Números racionais}

\begin{itemize}
\item O que acontece com os algoritmos da soma, subtração, multiplicação e divisão quando os números sendo operados não são inteiros?
\pause
\item Sem perder a generalidade, suporemos que $A$ e $B$ possuem $k$ algarismos depois da vírgula. (Se eles não tiverem a mesma quantidade de algarismos após a vírgula?)
\begin{eqnarray*}
A = a_{n-1} a_{n-2} \ldots a_1 a_0 & , & a_{-1} \ldots a_{-k} \\
B = b_{m-1} a_{m-2} \ldots b_0 & , & b_{-1} \ldots b_{-k}
\end{eqnarray*}
\vspace{-12pt}
\pause
\item Caso mais fácil: divisão. Note que:
$$
\frac{A}{B} = \frac{A \times 2^k}{B \times 2^k} = \frac{a_{n-1} a_{n-2} \ldots a_1 a_0 a_{-1} \ldots a_{-k}}{b_{m-1} a_{m-2} \ldots b_0 b_{-1} \ldots b_{-k}}
$$
\vspace{-12pt}
\pause
Simplesmente ignore a vírgula!
\end{itemize}

\end{frame}

\begin{frame}[fragile]
\frametitle{Números racionais}

\begin{itemize}
\item Para a soma e a subtração: como os algoritmos são ``copiados'' da versão para números na base $10$, a solução é simples: ignore, inicialmente a vírgula. Após a soma, recoloque a vírgula no seu lugar (conte $k$ algarismos à direita).
\pause
\item Para a multiplicação: de novo, a inspiração vem da base decimal. Ignore, inicialmente a vírgula e, após a multiplicação, recoloque a vírgula no seu lugar (conte $\mathbf{2k}$ algarismos à direita).

\end{itemize}

\textbf{Para casa}:
\begin{enumerate}[(1)]
\item escreva as versões dos algoritmos da soma, subtração, multiplicação e divisão para números racionais sem sinal (positivos) com $k$ algarismos após a vírgula;
\item altere os algoritmos para permitir números racionais com sinal.
\end{enumerate}

\end{frame}

\begin{frame}[fragile]
\frametitle{Representação numérica}

\begin{itemize}
\item Representação de números no papel: usamos tantos dígitos forem necessários.
\item Limitado apenas pela quantidade de papel, tempo disponível para escrever os dígitos, paciência\ldots
\end{itemize}

\pause

Número $\pi$:

\begin{Verbatim}[commandchars=\\\{\},codes={\catcode`$=3\catcode`^=7}]
3.14159265358979323846264338327950288419716939937510582097
4944592307816406286208998628034825342117067982148086513282
3066470938446095505822317253594081284811174502841027019385
2110555964462294895493038196442881097566593344612847564823
3786783165271201909145648566923460348610454326648213393607
2602491412737245870066063155881748815209209628292540917153
6436789259036001133053054882046652138414695194151160943305
7270365759591953092186117381932611793105118548074462379...
\end{Verbatim}

\end{frame}


\begin{frame}
\frametitle{Representação numérica num computador digital}

\begin{itemize}
\item Recordando: em um computador digital qualquer informação, em última instância, é representada por um número.
\pause
\item Atualmente, os números são representados internamente em binário (por vários motivos, entre eles facilidade de fazer contas na base $2$).
\pause
\item Um computador digital possui espaço finito para guardar informações.
\pause
\item Por questões de eficiência, geralmente o processamento de dados (ou seja, números) não é feito algarismo binário por algarismo binário, e sim \textbf{por grupos de algarismos binários} de uma só vez.
\end{itemize}

\pause


\end{frame}

\begin{frame}
\frametitle{Bits e palavras}

\begin{itemize}
\item Abreviação: algarismo binário = \textbf{bit} (do inglês \textbf{b}inary dig\textbf{it})
\pause
\item A unidade natural de processamento de um determinado sistema é chamada \textbf{palavra de dado}; é, basicamente, uma sequência de bits com tamanho fixo que é processada em conjunto.\\[6pt]
\hspace*{\fill}$MSB$\hspace{9em}$LSB$\hspace*{\fill}

\vspace{-3pt}

\hspace*{\fill}$\downarrow$\hspace{10.5em}$\downarrow$\hspace*{\fill}

\vspace{-16pt}

$$\underbrace{\fbox{\phantom{a}}\hspace{-0.5pt}\fbox{\phantom{a}}\hspace{-0.5pt}\fbox{\phantom{a}}\hspace{-0.5pt}\fbox{\phantom{a}}\hspace{-0.5pt}\fbox{\phantom{a}}\hspace{-0.5pt}\fbox{\phantom{a}}\hspace{-0.5pt}\fbox{\phantom{a}}\hspace{-0.5pt}\fbox{\phantom{a}}\hspace{-0.5pt}\fbox{\phantom{a}}\hspace{-0.5pt}\fbox{\phantom{a}}\hspace{-0.5pt}\fbox{\phantom{a}}}_{\text{tamanho } w = 11 \text{ bits}}$$\\
\comment{MSB = Most Significant Bit = bit mais significativo,\\LSB = Least Significant Bit = bit menos significativo}
\pause
\item Tamanhos de palavras comuns são: $4$, $8$, $16$, $32$ e $64$ bits.
\pause
\item Nomes comuns para palavras:
\begin{itemize}
\item $8$ bits = \textbf{byte} (\textbf{b}inar\textbf{y} \textbf{te}rm) ou \textbf{octeto}
\item $4$ bits = \textbf{nibble}\\\comment{\small(curiosidade: nibble, em inglês, significa ``mordidinha'' = ``small bite'')}
\end{itemize}
\end{itemize}

\end{frame}

\begin{frame}[fragile]
\frametitle{Representando números em palavras binárias}

\textbf{Primeiro caso}: número inteiro sem sinal ($\ge 0$).\\[12pt]

Como representar um número inteiro $A = (a_{n-1} a_{n-2} \ldots a_0)_2$ numa palavra de comprimento $w$?

\only<1>{
$$\underbrace{\fbox{\phantom{a}}\hspace{-0.5pt}\fbox{\phantom{a}}\hspace{-0.5pt}\fbox{\phantom{a}}\hspace{-0.5pt}\fbox{\phantom{a}}\hspace{-0.5pt}\fbox{\phantom{a}}\hspace{-0.5pt}\fbox{\phantom{a}}\ldots\fbox{\phantom{a}}\hspace{-0.5pt}\fbox{\phantom{a}}\hspace{-0.5pt}\fbox{\phantom{a}}\hspace{-0.5pt}\fbox{\phantom{a}}\hspace{-0.5pt}\fbox{\phantom{a}}}_{w \text{ bits}}$$
}%
\only<2->{
$$
\underbrace{\fbox{0}\hspace{-0.5pt}\fbox{0}\raisebox{2.5pt}{\hspace{-0.5pt}\fbox{$a_{n-1}$}\hspace{-0.5pt}\fbox{$a_{n-2}$}\ldots\fbox{$a_2$}\hspace{-0.5pt}\fbox{$a_1$}\hspace{-0.5pt}\fbox{$a_0$}}}_{w \text{ bits}}
$$
}%

\pause

Qual é o maior inteiro sem sinal que podemos representar?\\[12pt]

\pause

\textbf{Exemplo}: quais inteiros sem sinal podemos representar com $3$ bits?

\pause

\begin{Verbatim}[commandchars=\\\{\},codes={\catcode`$=3\catcode`^=7}]
000 001 010 011 100 101 110 111
 0   1   2   3   4   5   6   7
\end{Verbatim}

\pause

De $0$ até $7 = 2^3 - 1$.

\end{frame}

\begin{frame}[fragile]
\frametitle{Representando números: inteiros sem sinal}

Inteiros sem sinal em palavras binárias com $w$ bits.\\[12pt]

\begin{tabular}{rcl}
Palavra      &   &  Decimal \\
 $00\ldots000$ & = &  $0$ \\
 $00\ldots001$ & = &  $1$ \\
 $00\ldots010$ & = &  $2$ \\
               & \vdots & \\
 $11\ldots110$ & = &  \only<1-3>{?}\only<4->{$2^w - 2$} \\
 $11\ldots111$ & = &  \only<1-3>{?}\only<4->{$2^w - 1$} = maior inteiro sem sinal com $w$ bits \\
 \uncover<3->{$100\ldots000$} & \uncover<3->{=} & \uncover<3->{$2^w$}
\end{tabular}

\pause

\vspace{12pt}

O próximo número na sequência, que não cabe em $w$ bits, é
$$
 ({\color{red}1}\hspace{-1pt}\underbrace{{\color{mediumgray}0}0\ldots0{\color{pinegreen}0}{\color{blue}0}}_{w\text{ bits}})_2 = {\color{blue}0}\cdot2^0 + {\color{pinegreen}0}\cdot2^1 + \ldots + {\color{mediumgray}0} \cdot 2^{w-1} + {\color{red}1} \cdot 2^w = 2^w
$$

\end{frame}

\begin{frame}[fragile]
\frametitle{Representando números: inteiros com sinal}

\begin{itemize}
\item Precisamos reservar espaço na palavra para representar, além dos algarismos do número, alguma informação sobre o sinal.\pause
\item Como só existem duas possibilidades para o sinal, podemos usar um dos bits para representar o sinal. \pause Sugestão:
\begin{itemize}
\item sinal $+$: bit de sinal $0$
\item sinal $-$: bit de sinal $1$
\end{itemize}
$$
\underbrace{\overbrace{\fbox{$\,\,\,s\!\!\!\!\!\!\phantom{a_{w-1}}$}}^{sinal}\hspace{-2pt}\overbrace{\fbox{$a_{w-2}$}\hspace{-0.5pt}\fbox{$a_{w-1}$}\ldots\fbox{$a_2$}\hspace{-0.5pt}\fbox{$a_1$}\hspace{-0.5pt}\fbox{$a_0$}}^{magnitude}}_{w \text{ bits}}
$$
\item Esta representação é conhecida como \textbf{sinal-magnitude}.
\pause
\item \textbf{Ex.:} inteiros representados em sinal-magnitude com $3$ bits
\begin{Verbatim}[commandchars=\\\{\},codes={\catcode`$=3\catcode`^=7}]
000 001 010 011 100 101 110 111
 +0  +1  +2  +3  -0  -1  -2  -3
\end{Verbatim}
\end{itemize}
\end{frame}


\begin{frame}[fragile]
\frametitle{Representação sinal-magnitude}

\textbf{Menor número}: \WD{1}\WD{1}\WD{1}\ldots\WD{1}\WD{1} = \pause $-(\underbrace{11\ldots11}_{w-1\text{ uns}})_2$ = \pause $-(1\underbrace{00\ldots00}_{w-1\text{ zeros}} - 1)_2$\\[-6pt] = $-(2^{w-1} - 1) = -2^{w-1} + 1$\\[12pt]

\pause

\textbf{Maior número}: \WD{0}\WD{1}\WD{1}\ldots\WD{1}\WD{1} = +$2^{w-1} - 1$\\[12pt]

\pause

\textbf{Vantagens}

\begin{itemize}
\item simples de entender
\item simples de implementar
\end{itemize}

\textbf{Desvantagens}

\begin{itemize}
\item zero tem duas representações: \WD{0}\WD{0}\ldots\WD{0} = $+0$ e \WD{1}\WD{0}\ldots\WD{0} = $-0$
\item complica a aritmética: é necessário tratar o sinal separadamente na hora de fazer as contas de soma e subtração.
\end{itemize}

\end{frame}

\begin{frame}
\frametitle{Representação em complemento de $2$}

\begin{itemize}
\item vimos que, uma maneira de fazer subtrações na forma $A - B$ era tomar o complemento a dois $\overline{B} + 1$ e fazer a soma $A + (\overline{B} + 1)$
\item note que se $-B$ é um número negativo, então $-B = 0-B$
\item suponha que estamos representando todos os números positivos em palavras binárias de tamanho $w$ na forma:
$$
\WD{\,0\,}\raisebox{2.5pt}{\WD{$a_{w-2}$}\WD{$a_{w-3}$}\ldots\WD{$a_1$}\WD{$a_0$}}
$$
\pause
\item \textbf{Ex.:} Calcule $0 - (11)_{10}$ usando complemento de $2$ em palavras com $5$ bits, sendo que o primeiro bit $0$ representa sinal positivo.\\
\hspace{2cm}
$0$ = $\WD{0}\hspace{-1.5pt}\underbrace{\WD{0}\WD{0}\WD{0}\WD{0}}_{0000_2 = 0}$
\hspace{1cm}
$11_{10}$ = $\WD{0}\hspace{-1.5pt}\underbrace{\WD{1}\WD{0}\WD{1}\WD{1}}_{1011_2 = 11_{10}}$\\[6pt]\pause
Complemento a dois de $01011$ = $10100 + 1$ = $10101$\\[6pt]\pause
$-1011_2 = 0 - 1011_2 = 0 + 10101_2 = \WD{1}\WD{0}\WD{1}\WD{0}\WD{1}$, bit de sinal $1$
\end{itemize}

\end{frame}

\begin{frame}
\frametitle{Representação em complemento de $2$}

Representação de inteiros com sinal em complemento de $2$

\begin{itemize}
\item Números positivos
$$
    \raisebox{-2.5pt}{\WD{0}}\hspace{-1.8pt}\underbrace{\WD{$a_{w-2}$}\WD{$a_{w-3}$}\ldots\WD{$a_1$}\WD{$a_0$}}_{w-1\text{ bits}} = + (a_{w-2} a_{w-3} \ldots a_1 a_0)_2
$$
\item Números negativos
$$
    \raisebox{-2.5pt}{\WD{1}}\hspace{-1.8pt}\underbrace{\WD{$a_{w-2}$}\WD{$a_{w-3}$}\ldots\WD{$a_1$}\WD{$a_0$}}_{w-1\text{ bits}} = - (\overline{a_{w-2} a_{w-3} \ldots a_1 a_0} + 1)_2
$$
\item \textbf{Ex.:} a que número corresponde a palavra \WD{1}\WD{0}\WD{1}\WD{1}\WD{1}\WD{0}\WD{0} ?\\
\pause
Bit de sinal $1$ = número negativo.\\
\pause
$\overline{\texttt{011100}}$ + 1 = \texttt{100011} + 1 = \texttt{100100} = $36_{10}$\\
\pause
\WD{1}\WD{0}\WD{1}\WD{1}\WD{1}\WD{0}\WD{0} = $-36_{10}$
\end{itemize}

\end{frame}

\begin{frame}[fragile]
\frametitle{Representação em complemento de $2$}

Inteiros representados em complemento de dois em palavras de $3$ bits:

\texttt{011} = $+3_{10}$\\
\texttt{010} = $+2_{10}$\\
\texttt{001} = $+1_{10}$\\
\texttt{000} = $0_{10}$\\
\pause
\texttt{111} = $\uncover<3->{-1_{10} \hspace{5mm} =} -(\overline{\texttt{11}} + 1)_2 = -(00 + 1)_2$\\
\pause
\pause
\texttt{110} = $\uncover<5->{-2_{10} \hspace{5mm} =} -(\overline{\texttt{10}} + 1)_2 = -(01 + 1)_2$\\
\pause
\pause
\texttt{101} = $-3_{10}$\\
\pause
\texttt{100} = $-4_{10}$\\

\pause

\begin{itemize}
\item note que o intervalo de representação não é simétrico
\item como só há uma representação para $0$, é possível representar um inteiro negativo a mais
\item somas/subtrações com esta representação são simples!\\
$1 + (-3)$ = \WD{0}\WD{0}\WD{1} + \WD{1}\WD{0}\WD{1} = \WD{1}\WD{1}\WD{0} = $-2$
\end{itemize}

\end{frame}

\begin{frame}[fragile]
\frametitle{Representação em complemento de $2$}

\textbf{Maior número}: \WD{0}\WD{1}\WD{1}\ldots\WD{1}\WD{1} = $+2^{w-1} - 1$ (como sinal-magnitude)\\[12pt]

\pause

\textbf{Menor número}: \WD{1}\WD{0}\WD{0}\ldots\WD{0}\WD{0} = $-(\overline{00\ldots00} + 1)_2 = - (11\ldots11 + 1)$\\[6pt]
\pause
\hfill$= - (1\underbrace{00\ldots00}_{w-1\text{ zeros}} - 1 + 1) \pause = -2^{w-1}$\\(uma unidade menor que sinal-magnitude)

\pause

\textbf{Vantagens}

\begin{itemize}
\item representação única para o zero
\item somas e subtrações são feitas da mesma forma que para números sem sinal
\end{itemize}

\textbf{Desvantagens}

\begin{itemize}
\item não é tão intuitivo para nós (indiferente para computador)
\item comparação não é tão simples: $1 = \WD{0}\WD{0}\WD{1} > \WD{1}\WD{0}\WD{1} -3$
\end{itemize}

\end{frame}

\begin{frame}
\frametitle{Para casa}

\begin{itemize}
\item Pensar:
\begin{itemize}
\item como converter palavras de dados de tamanhos diferentes? Ex.: de 8 para 16 bits?
\pause
\item O que acontece se o resultado da soma/subtração/multiplicação de dois inteiros representados em palavras de $w$ bits não couber em $w$ bits? (Overflow)
\pause
\item Comportamentos distintos para representação sem sinal, sinal-magnitude e complemento de dois
\pause
\end{itemize}
\item Seções do livro: 2-4, 2-5, 2-6 e 2-7
\end{itemize}

\end{frame}


\end{document}
