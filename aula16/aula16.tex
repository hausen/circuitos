\documentclass[a4paper,12pt,notitlepage]{article}
\usepackage[T1]{fontenc}
\usepackage[utf8]{inputenc}
\usepackage[portuguese,brazil]{babel}
\usepackage[left=2cm,right=2cm,top=2cm,bottom=2cm]{geometry}
\usepackage{eulervm,palatino}
\usepackage{amsmath,amssymb}
\usepackage[table]{xcolor}
\usepackage{tikz}
\usepackage{ifthen}

\usetikzlibrary{automata,positioning,calc,decorations.markings}
\usetikzlibrary{arrows,shapes.gates.logic.US,shapes.gates.logic.IEC,calc}

\newenvironment{absolutelynopagebreak}
  {\par\nobreak\vfil\penalty0\vfilneg
   \vtop\bgroup}
  {\par\xdef\tpd{\the\prevdepth}\egroup
   \prevdepth=\tpd}

\title{Registradores de Deslocamento e Memórias}
\author{Rodrigo Hausen}
\date{}

\newtheorem{exer}{Exercício}
\newenvironment{sol}{\noindent\textbf{Solução} }{\hfill$\blacksquare$}

\newcommand{\parasabermais}[1]
{%
\noindent\hspace*{-2\fboxsep}%
\fcolorbox{black!20}{black!20}{%
\begin{minipage}{\textwidth}
\begin{center}
\textbf{Para saber mais}
\end{center}
#1
\end{minipage}
}}


\newcommand*{\Dff}[3]{\draw[thick] (#1,#2-1.5) rectangle ++(2,2);
  \node (#3d) at (#1+0.3,#2) {$D$};
  \node (#3q) at (#1+1.7,#2) {$Q$};
  \node (#3nq) at (#1+1.7,#2-1) {$\overline{Q}$};
  \coordinate (#3ck) at (#1,#2-1);
  \draw (#1+0.2,#2-0.8) -- (#1+0.4,#2-0.8) -- (#1+0.4,#2-1.2) -- (#1+0.6,#2-1.2);
  \draw[->] (#1+0.4,#2-0.8) -- (#1+0.4,#2-1.02);
}

\newcommand*{\muxtwor}[3]{%
  \node[anchor=west] (#3i0) at (#1,#2) {{\small$I_0$}};
  \node[anchor=west] (#3i1) at (#1,#2+1) {{\small$I_1$}};
  \draw[thick] (#3i0.west) -- ++(0,-0.5) -- ($(#3i0.west)+(1,0)$)
               -- ($(#3i1.west)+(1,0)$) -- ($(#3i1.west)+(0,0.5)$)
               -- (#3i0.west);
  \coordinate (#3sel) at ($(#3i1.west)+(0.5,0.25)$);
  \coordinate (#3out) at ($(#3i0.west)!0.5!(#3i1.west)+(1,0)$);
}

\newcommand*{\branch}[3]{\node[mark size=1.5pt,inner sep=-2pt] (#3) at (#1-0.05,#2) {\pgfuseplotmark{*}};}


\begin{document}
\maketitle

\section{Registradores de deslocamento}

\begin{exer}
Usando \emph{flip-flops} do tipo D, projete uma máquina de estado com duas
entradas, $Ck$ (\emph{clock}) e $d$ (um bit de dado), e duas saídas, $Q_1$,
$Q_0$, tais que a cada borda de descida do clock as saídas mudam de acordo
com a seguinte tabela de transição:
\begin{center}
\begin{tabular}{cc||cc}
\multicolumn{2}{r||}{estado atual} & \multicolumn{2}{l}{próx. estado} \\
$Q_1$ & $Q_0$ & $Y_1$ & $Y_0$ \\
\hline
$a_1$ & $a_0$ & $d$   & $a_1$
\end{tabular}
\end{center}
\end{exer}

\begin{sol}
\emph{Passo um}: diagrama de estados.
\begin{center}
\begin{tikzpicture}[shorten >=1pt,node distance=2.8cm,on grid,auto]
  \node[state] (00)   {$00$};
  \node[state] (10) [right=of 00]  {$10$};
  \node[state] (01) [below=of 10]  {$01$};
  \node[state] (11) [right=of 10]  {$11$};
  \path[->] (00) edge [loop above] node {$d=0$} (); 
  \path[->] (00) edge node {$d=1$} (10); 
  \path[->] (10) edge [bend left=15] node {$d=0$} (01); 
  \path[->] (10) edge node {$d=1$} (11); 
  \path[->] (11) edge [bend left=30] node {$d=0$} (01); 
  \path[->] (11) edge [loop above] node {$d=1$} (); 
  \path[->] (01) edge [bend left=30] node {$d=0$} (00); 
  \path[->] (01) edge [bend left=15] node {$d=1$} (10); 
\end{tikzpicture}
\end{center}

\def\ca{\cellcolor{blue!15}}
\def\cb{\cellcolor{red!15}}

\emph{Passo dois}: tabelas verdade.
\begin{center}
\begin{tabular}{ccc||cc}
$d$ & $Q_1$ & $Q_0$ & $Y_1$ & $Y_0$ \\
\hline
\ca 0 & \cb 0 & 0 & \ca 0 & \cb 0 \\
\ca 0 & \cb 0 & 1 & \ca 0 & \cb 0 \\
\ca 0 & \cb 1 & 0 & \ca 0 & \cb 1 \\
\ca 0 & \cb 1 & 1 & \ca 0 & \cb 1 \\
\ca 1 & \cb 0 & 0 & \ca 1 & \cb 0 \\
\ca 1 & \cb 0 & 1 & \ca 1 & \cb 0 \\
\ca 1 & \cb 1 & 0 & \ca 1 & \cb 1 \\
\ca 1 & \cb 1 & 1 & \ca 1 & \cb 1
\end{tabular}
\end{center}

\emph{Passo três}: expressões para $Y_0$ e $Y_1$.

Pela observação das tabelas verdade, verificamos imediatamente que:
\begin{eqnarray*}
  Y_1 & = & d \\
  Y_0 & = & Q_1
\end{eqnarray*}

\begin{absolutelynopagebreak}
\emph{Passo quatro}: diagrama do circuito.
\begin{center}
\begin{tikzpicture}
%\draw[help lines, thick] (0,0) grid (8,3);
\node[anchor=east] (d) at (0,2) {$d$};
\node[anchor=east] (Ck) at (0,0) {$Ck$};
\node[anchor=west] (Q1) at (8,3) {$Q_1$};
\node[anchor=west] (Q0) at (8,2) {$Q_0$};

\branch{0.5}{0}{branch1}
\branch{4}{2}{branch2}
\Dff{1}{2}{dff1}
\Dff{5}{2}{dff2}
\path[->] (d) edge node [below] {$Y_1$} (dff1d);
\path[->] (Ck) edge node {} (branch1);
\draw[-] (0,0) -- (0.5,0) -- (0.5,1) -- (1,1);
\draw[-] (0,0) -- (4,0) -- (4,1) -- (5,1);
\path[-] (dff1q) edge node [below] {$Y_0$} (dff2d);
\draw[->] (4,2) -- (4,3) -- (8,3);
\path[->] (dff2q) edge node {} (Q0);
\end{tikzpicture}
\end{center}
\end{absolutelynopagebreak}
\end{sol}

O circuito acima é chamado \emph{registrador de deslocamento} de $2$ bits
\emph{com entrada serial e saída paralela} (ou, simplesmente, \emph{série
para paralelo}).

\begin{exer}
Abaixo temos o circuito para um registrador de deslocamento série para
paralelo de $4$ bits:
\begin{center}
\tikzstyle{branch}=[fill,shape=circle,minimum size=3pt,inner sep=0pt]
\begin{tikzpicture}
%\draw[help lines, thick] (0,0) grid (13,5);
\node[anchor=east] (d) at (0,2) {$d$};
\node[anchor=east] (Ck) at (0,0) {$Ck$};
\node (Q0) at (13,2) {$Q_0$};
\node (Q1) at (13,2.8) {$Q_1$};
\node (Q2) at (13,3.6) {$Q_2$};
\node (Q3) at (13,4.4) {$Q_3$};
\Dff{1}{2}{dff1}
\Dff{4}{2}{dff2}
\Dff{7}{2}{dff3}
\Dff{10}{2}{dff4}

\path (Ck) -- coordinate (branch1) (Ck -| dff1ck);
\draw (branch1) node[branch] {} |- (dff1ck);
\path (Ck) -- coordinate (branch2) (Ck -| dff2ck);
\draw (branch2)+(1.5,0) node[branch] {} |- (dff2ck);
\path (Ck) -- coordinate (branch3) (Ck -| dff3ck);
\draw (branch3)+(3,0) node[branch] {} |- (dff3ck);
\path (Ck) -- coordinate (branch4) (Ck -| dff4ck);
\draw (branch4)+(4.5,0) node {} |- (dff4ck);
\draw (Ck) -- (9.5,0);

\draw (d) -- (dff1d);

\draw (dff1q) -- (dff2d);
\node[branch] (branch5) at ($(dff1q)!0.5!(dff2d)$) {};
\draw (branch5) |- (Q3);

\draw (dff2q) -- (dff3d);
\node[branch] (branch6) at ($(dff2q)!0.5!(dff3d)$) {};
\draw (branch6) |- (Q2);

\draw (dff3q) -- (dff4d);
\node[branch] (branch7) at ($(dff3q)!0.5!(dff4d)$) {};
\draw (branch7) |- (Q1);

\draw (dff4q) -- (Q0);
\end{tikzpicture}
\end{center}
\end{exer}

Considerando os diagramas de forma de onda para $d$ e $Ck$, e que o estado
inicial de $Q_0$, $Q_1$, $Q_2$ e $Q_3$ é zero, esboce os diagramas de forma
de onda para as saídas $Q_0$, $Q_1$, $Q_2$ e $Q_3$.

\newcommand{\clockcycle}[2]{\draw[very thick] (#1,#2) -- (#1+1,#2) -- (#1+1,#2+1) -- (#1+2,#2+1) -- (#1+2,#2) -- (#1+3,#2); \draw[->,very thick] (#1+2,#2+1) -- (#1+2,#2+0.5);}

\begin{center}
\begin{tikzpicture}
%\draw[help lines, thick] (0,0) grid (13,9);

\node[anchor=east] (Ck) at (0,8.5) {$Ck$};
\foreach \i in {0,2,4,6,8,10}
{
  \draw[dashed] (\i+2,0) -- (\i+2,9);
  \clockcycle{\i}{8}
}

\node[anchor=east] (d) at (0,7.0) {$d$};
\draw[very thick] (0,6.5) -- (3,6.5) -- (3,7.5) -- (5,7.5) -- (5,6.5) -- (13,6.5);

\node (saidas) at (5,6) {Saídas:};

\node[anchor=east] (Q3) at (0,5.0) {$Q_3$};
\draw[very thick] (0,4.5)--(4,4.5)--(4,5.5)--(6,5.5)--(6,4.5)--(13,4.5);

\node[anchor=east] (Q2) at (0,3.5) {$Q_2$};
\draw[very thick] (0,3.0)--(6,3.0)--(6,4.0)--(8,4.0)--(8,3.0)--(13,3.0);

\node[anchor=east] (Q1) at (0,2.0) {$Q_1$};
\draw[very thick] (0,1.5)--(8,1.5)--(8,2.5)--(10,2.5)--(10,1.5)--(13,1.5);

\node[anchor=east] (Q0) at (0,0.5) {$Q_0$};
\draw[very thick] (0,0)--(10,0)--(10,1)--(12,1)--(12,0)--(13,0);
\end{tikzpicture}
\end{center}

\begin{exer}
\begin{absolutelynopagebreak}
Circuito para um registrador de deslocamento paralelo para
série de $4$ bits:

\noindent
\resizebox{\textwidth}{!}{
\tikzstyle{branch}=[fill,shape=circle,minimum size=3pt,inner sep=0pt]
\begin{tikzpicture}
\Dff{1}{2}{dff1}
\Dff{5}{2}{dff2}
\Dff{9}{2}{dff3}
\Dff{13}{2}{dff4}
\muxtwor{3.5}{2}{mux1}
\muxtwor{7.5}{2}{mux2}
\muxtwor{11.5}{2}{mux3}

\node[thick,and gate US,draw,anchor=output,logic gate inputs=nn,rotate=-90]
  (andd0) at ($(dff1d.west)+(-0.5,0.5)$) {};

\node[anchor=north] (d0) at (andd0.input 1 |- 0.5,5.5) {$d_3$};
\node[anchor=north] (d1) at (3,5.5) {$d_2$};
\node[anchor=north] (d2) at (7,5.5) {$d_1$};
\node[anchor=north] (d3) at (11,5.5) {$d_0$};

\draw(d0) |- (andd0.input 1);
\draw(andd0) |- (dff1d);
\draw(d1) |- (mux1i1);
\draw(d2) |- (mux2i1);
\draw(d3) |- (mux3i1);

\draw (dff1q) -- (mux1i0);
\draw (mux1out) -| ($(dff2d.west)!0.5!(mux1out)$) |- (dff2d.west);

\draw (dff2q) -- (mux2i0);
\draw (mux2out) -| ($(dff3d.west)!0.5!(mux2out)$) |- (dff3d.west);

\draw (dff3q) -- (mux3i0);
\draw (mux3out) -| ($(dff4d.west)!0.5!(mux3out)$) |- (dff4d.west);

\node[anchor=east] (Ck) at (-0.5,0) {$Ck$};
\draw (Ck) -- coordinate(branch1) (Ck -| d0) node[branch] {} |- (dff1ck);
\draw (branch1) -- coordinate(branch2) (branch1 -| mux1sel) node[branch] {}
      |- (dff2ck);
\draw (branch2) -- coordinate(branch3) (branch2 -| mux2sel) node[branch] {}
      |- (dff3ck);
\draw (branch3) -- coordinate(branch4) (branch3 -| mux3sel)
      |- (dff4ck);

\node[anchor=east] (ldsh) at (-0.5,4) {$\overline{SH}/LD$};
\draw (ldsh) -| node[branch] {} (andd0.input 2);
\draw (ldsh) -- coordinate (branch5) (ldsh -| mux1sel) node[branch] {}
      -- (mux1sel);
\draw (branch5) -- coordinate (branch6) (branch5 -| mux2sel) node[branch] {}
      -- (mux2sel);
\draw (branch5) -| (mux3sel);

\draw (dff4q) -- ++(1,0) node[anchor=west] {$d_{out}$};

\end{tikzpicture}
}% end resizebox
\end{absolutelynopagebreak}

Neste registrador de deslocamento, a entrada $\overline{SH}/LD$
(``\emph{SHift or LoaD}'')  indica qual operação será realizada:
\begin{itemize}
  \item se $\overline{SH}/LD = 1$ será feito o carregamento (\emph{load})
        dos bits de dados no registrador: na próxima borda de descida do
        clock, os bits $d_3 \ldots d_0$ são armazenados nos flip-flops
        e o bit $d_0$ será colocado na saída $d_{out}$.
  \item se $\overline{SH}/LD = 0$ será feito o deslocamento (\emph{shift})
        dos bits para a saída: a cada borda negativa do clock, um
        bit de dado será colocado na saída do clock, na sequência
        $d_1, d_2, d_3, 0, 0, \ldots$.

Esboce o diagrama de forma de onda para o carregamento do dado
$d_3 d_2 d_1 d_0 = 1011$ e seu deslocamento para a saída.
\end{itemize}
\end{exer}

\begin{center}
\begin{tikzpicture}
%\draw[help lines, thick] (0,0) grid (13,9);

\node[anchor=east] (Ck) at (0,8.5) {$Ck$};
\foreach \i in {0,2,4,6,8,10}
{
  \draw[dashed] (\i+2,-1.5) -- (\i+2,9);
  \clockcycle{\i}{8}
}

\node[anchor=east] (shld) at (0,7.0) {$\overline{SH}/LD$};
\draw[very thick] (0,6.5) -- ++(1,0) -- ++(0,1) -- ++(2,0)
     -- ++(0,-1) -- ++(10,0);

\node[anchor=east] (d3) at (0,5.5) {$d_3$};
\draw[very thick] (0,5) -- ++(1,0) -- ++(0,1) -- coordinate(d3in) ++(2,0)
     -- ++(0,-1) -- ++(10,0);
\node at (d3 -| d3in) {\textbf{1}};

\node[anchor=east] (d2) at (0,4.0) {$d_2$};
\draw[very thick] (0,3.5) -- ++(1,0) -- ++(0,0) -- coordinate(d2in) ++(2,0)
     -- ++(0,0) -- ++(10,0);
\node at (d2 -| d2in) {\textbf{0}};

\node[anchor=east] (d1) at (0,2.5) {$d_1$};
\draw[very thick] (0,2) -- ++(1,0) -- ++(0,1) -- coordinate(d1in) ++(2,0)
     -- ++(0,-1) -- ++(10,0);
\node at (d1 -| d1in) {\textbf{1}};

\node[anchor=east] (d0) at (0,1.0) {$d_0$};
\draw[very thick] (0,.5) -- ++(1,0) -- ++(0,1) -- coordinate(d0in) ++(2,0)
     -- ++(0,-1) -- ++(10,0);
\node at (d0 -| d0in) {\textbf{1}};

\node at (2,0) {Saída:};

\node[anchor=east] (dout) at (0,-1.0) {$d_{out}$};
\draw[very thick] (0,-1.5) -- ++(2,0) -- ++(0,1) -- coordinate(d0out) ++(2,0)
     -- ++(0,0) -- coordinate(d1out) ++(2,0) -- ++(0,-1)
     -- coordinate(d2out) ++(2,0) -- ++(0,1) -- coordinate(d3out) ++(2,0)
     -- ++(0,-1) -- ++(3,0);

\node at (dout -| d0out) {\textbf{1}};
\node at (dout -| d1out) {\textbf{1}};
\node at (dout -| d2out) {\textbf{0}};
\node at (dout -| d3out) {\textbf{1}};

\end{tikzpicture}
\end{center}

\newpage

\newcommand{\stereobox}[5][]{
  \draw[thick] (#2,#3) rectangle ++(#4-0.2,#5-0.2) node[pos=.5] {#1};
  \draw[thick] (#2+#4-0.2,#3) -- (#2+#4+0.2,#3+0.4) -- (#2+#4+0.2,#3+#5+0.2)
               -- (#2+0.4,#3+#5+0.2) -- (#2,#3+#5-0.2);
  \draw[thick] (#2+#4+0.2,#3+#5+0.2) -- (#2+#4-0.2,#3+#5-0.2);
}

\newcommand{\cknode}[3]{
  \node[matrix,anchor=west] (#1) at (#2,#3) {\draw[-stealth] (0,0.2) -- (0.2,0.2) -- (0.2,-0.05); \draw[-] (0.2,0) -- (0.2,-0.2) -- (0.4,-0.2); \\ };
}


\subsection{Para que servem registradores de deslocamento?}

\textbf{Estudo de caso de uso 1:} controles do NES (Nintendo Entertainment System) e do SNES (Super Nintendo Entertainment System).

O controle padrão do NES possui 8 botões ($\leftarrow$, $\uparrow$,
$\rightarrow$, $\downarrow$, A, B, select, start) e o do SNES possui 12
(além dos botões do NES, possui X, Y, L e R). Todos os botões podem ser
acionados simultaneamente e o console de vídeo-game verifica o estado
deles 60 vezes por segundo.

Para evitar o uso de um cabo com 8 ou 12 fios (isso sem contar com os fios
de alimentação!), os controles possuem um registrador de deslocamento paralelo
para série, cujas entradas de dados são o estado dos botões. O sinal
de controle $\overline{SH}/LD$ e o clock vêm do console, enquanto que a saída
$d_{out}$ do registrador de deslocamento é retornada do controle para o
console.

Com a estratégia de transmissão serial do estado dos botões, o cabo que conecta
os controles ao console precisa de apenas $5$ fios ($\overline{SH}/LD$, $Ck$,
$d_{out}$, +5V e GND/terra), não importa quantos botões o controle tenha!
Quanto menos fios um cabo possui, mais barato e flexível ele é, além de ser
menos propenso a falhas.

Apresentamos abaixo o esquema de ligação interna entre os componentes do
controle do NES.

\begin{center}
\begin{tikzpicture}
\def\txsize{\scriptsize}
\tikzset{>=triangle 45}
\node[anchor=east,align=right] (shld) at (-1,2)
     {$\overline{SH}/LD$\\[-0.3\baselineskip]{\txsize(laranja)}};
\node[anchor=east,align=right] (ck) at (-1,0)
     {$Ck$\\[-0.3\baselineskip]{\txsize(vermelho)}};

\node[anchor=north,align=center] (d7) at (1.5,3)
     {{\txsize(7)}\\$d_7$};
\draw[<-] (d7.north) -- ++(0,1) node[anchor=south] {$\rightarrow$};
\node[anchor=north,align=center] (d6) at ($(d7.north)+(0.75,0)$)
     {{\txsize(6)}\\$d_6$};
\draw[<-] (d6.north) -- ++(0,1) node[anchor=south] {$\leftarrow$};
\node[anchor=north,align=center] (d5) at ($(d6.north)+(0.75,0)$)
     {{\txsize(5)}\\$d_5$};
\draw[<-] (d5.north) -- ++(0,1) node[anchor=south] {$\downarrow$};
\node[anchor=north,align=center] (d4) at ($(d5.north)+(0.75,0)$)
     {{\txsize(4)}\\$d_4$};
\draw[<-] (d4.north) -- ++(0,1) node[anchor=south] {$\uparrow$};
\node[anchor=north,align=center] (d3) at ($(d4.north)+(0.75,0)$)
     {{\txsize(13)}\\$d_3$};
\draw[<-] (d3.north) -- ++(0,1) node[rotate=45,anchor=south west] {start};
\node[anchor=north,align=center] (d2) at ($(d3.north)+(0.75,0)$)
     {{\txsize(14)}\\$d_2$};
\draw[<-] (d2.north) -- ++(0,1) node[rotate=45,anchor=south west] {select};
\node[anchor=north,align=center] (d1) at ($(d2.north)+(0.75,0)$)
     {{\txsize(15)}\\$d_1$};
\draw[<-] (d1.north) -- ++(0,1) node[anchor=south] {$B$};
\node[anchor=north,align=center] (d0) at ($(d1.north)+(0.75,0)$)
     {{\txsize(1)}\\$d_0$};
\draw[<-] (d0.north) -- ++(0,1) node[anchor=south] {$A$};
\draw[->] (shld.east) -- ++(1,0) node[anchor=west] (9) {{\txsize(9)}};
\draw[->] (ck.east) -- ++(1,0) node[anchor=west] (10) {{\txsize(10)}};
\draw[-{stealth}] ($(10.east)+(0,-0.25)$) -- ++(0.2,0) --
     coordinate(ckarrow) ++(0,0.25) ;
\draw(ckarrow) -- ++(0,0.25) -- ++(0.2,0);

\coordinate (bottomleft) at (0,-1);

\draw (bottomleft) -- (0,0 |- d0.north) -- (d0.north)
      -- ++(1.5,0) coordinate(topright) |- (bottomleft);

\coordinate (bottomright) at (bottomleft -| topright);

\node[align=center,yshift=-\baselineskip,xshift=1em]
     (4021) at ($(bottomleft)!0.5!(topright)$)
     {reg. de deslocamento \\ CD4021};

\node[anchor=east] (dout) at (topright |- shld) {\txsize(3)};

\draw[->] (dout.east) -- ++(1,0) node[anchor=west,align=left]
                                 {$d_{out}$\\[-0.3\baselineskip]
                                  {\txsize(amarelo)}};

\node[anchor=south] (vcc) at (bottomleft -| d4) {\txsize(16)};
\node[anchor=south] (vdd) at (bottomleft -| d2) {\txsize(8)};

\draw[<-] (vcc) -- ++(0,-1)
                   node[anchor=north,align=center]
                   {+5V \\[-0.3\baselineskip] {\txsize(branco)}};

\draw[<-] (vdd) -- ++(0,-1)
                   node[anchor=north,align=center]
                   {GND \\[-0.3\baselineskip] {\txsize(preto)}};

\end{tikzpicture}
\end{center}

\textbf{Estudo de caso de uso 2:} suponha que você possua um
processador/micro\-con\-tro\-la\-dor/Ar\-du\-ino com apenas $3$ saídas disponíveis e
deseja controlar $n > 3$ dispositivos diferentes $d_0$, $d_1$, \ldots, $d_{n-1}$.

\begin{center}
\begin{tikzpicture}
\tikzset{>=triangle 45}
%\draw[help lines, thick] (0,0) grid (14,6);

\draw[thick] (0.5,1) -- (5.5,1) -- (5.5,6) -- (0.5,6);
\draw[thick,dotted] (0,1) -- (0.5,1);
\draw[thick,dotted] (0,6) -- (0.5,6);
\node[yshift=\baselineskip] at (2.5,3.5) {CPU ou};
\node at (2.5,3.5) {Microcontrolador};
\node[yshift=-\baselineskip] at (2.5,3.5) {ou Arduino};
\node[anchor=east] (X) at (5.5,5.5) {$X$};
\node[anchor=east] (Y) at (5.5,4.5) {$Y$};
\node[anchor=east] (Z) at (5.5,2) {$Z$};

\cknode{ckshiftreg}{7.5}{4.5};
\node[anchor=west] (dshiftreg) at (7.5,5.5) {$D$};
\draw[thick] (7.5,4) rectangle ++(6,2)
             node[pos=.5,yshift=\baselineskip,xshift=1em]
               {\small Registrador de deslocamento}
             node[pos=.5,xshift=1em] {\small série para paralelo $n$ bits};
\node[anchor=south] (qn1) at (9,4) {\footnotesize$Q_{n-1}$};
\node[anchor=south] (qn2) at (10,4) {\footnotesize$Q_{n-2}$};
\node[anchor=south] (q1) at (12,4) {\footnotesize$Q_{1}$};
\node[anchor=south] (q0) at (13,4) {\footnotesize$Q_{0}$};

\draw[->] (qn1) -- (9,3);
\draw[->] (qn2) -- (10,3);
\draw[->] (q1) -- (12,3);
\draw[->] (q0) -- (13,3);

\cknode{cksyncreg}{7.5}{2};
\stereobox[\hspace*{2em}{\small Registrador síncrono $n$ bits}]{7.5}{1}{6}{2};

\node (dn1) at (9,0) {$d_{n-1}$};
\node (dn2) at (10,0) {$d_{n-2}$};
\node (ellip) at (11,0) {$\ldots$};
\node (d1) at (12,0) {$d_1$};
\node (d0) at (13,0) {$d_0$};

\draw[->] (dn1)+(0,1) -- (dn1);
\draw[->] (dn2)+(0,1) -- (dn2);
\draw[->] (d1)+(0,1) -- (d1);
\draw[->] (d0)+(0,1) -- (d0);

\draw[->] (X) -- (dshiftreg);
\draw[->] (Y) -- (ckshiftreg);
\draw[->] (Z) -- (cksyncreg);

\end{tikzpicture}
\end{center}

\newpage

\emph{Exemplo:} diagramas de forma de onda para fazer $d_0 = 1$, $d_1 = 1$, $d_2 = 0$,
$d_3 = 1$ (ou seja, $d_3 d_2 d_1 d_0 = 1 0 1 1$).

\begin{center}
\begin{tikzpicture}
%\draw[help lines, thick] (0,0) grid (12,16);

% forma de onda de Y
\node[anchor=east] (Y) at (0,15.5) {$Y$};
\foreach \i in {0,2,4,6}
{
  \draw[dashed] (\i+2,6) -- (\i+2,16);
  \clockcycle{\i}{15}
}
\draw[very thick](9,15)--(12,15);

% forma de onda de X
\node[anchor=east] (X) at (0,14) {$X$};
\draw[very thick](0,13.5)--(1,13.5)--(1,14.5)--(5,14.5)--(5,13.5)--
                 (7,13.5)--(7,14.5)--(9,14.5)--(9,13.5)--(12,13.5);
\node at (2,14) {\textbf{1}}; \node at (4,14) {\textbf{1}};
\node at (6,14) {\textbf{0}}; \node at (8,14) {\textbf{1}};

% forma de onda de Z
\node[anchor=east] (Z) at (0,12.5) {$Z$};
\draw[dashed] (9,13) -- (9,0);
\draw[very thick](0,12)--(8,12)--(8,13)--(9,13)--(9,12)--(12,12);
\draw[->,very thick] (9,13)--(9,12.5);


% forma de onda de Q3
\node[anchor=east] (q3) at (0,11) {$Q_3$};
\node at ($(q3)+(9,0)$) {\textbf{1}};
\draw[very thick](0,10.5)--(2,10.5)--(2,11.5)--(6,11.5)--(6,10.5)--
                 (8,10.5)--(8,11.5)--(12,11.5);

% forma de onda de Q2
\node[anchor=east] (q2) at (0,9.5) {$Q_2$};
\node at ($(q2)+(9,0)$) {\textbf{0}};
\draw[very thick](0,9)--(4,9)--(4,10)--(8,10)--(8,9)--(12,9);

% forma de onda de Q1
\node[anchor=east] (q1) at (0,8) {$Q_1$};
\node at ($(q1)+(9,0)$) {\textbf{1}};
\draw[very thick](0,7.5)--(6,7.5)--(6,8.5)--(12,8.5);

% forma de onda de Q0
\node[anchor=east] (q0) at (0,6.5) {$Q_0$};
\node at ($(q0)+(9,0)$) {\textbf{1}};
\draw[very thick](0,6)--(8,6)--(8,7)--(12,7);

% forma de onda de d3
\node[anchor=east] (d3) at (0,5) {$d_3$};
\node at ($(d3)+(10,0)$) {\textbf{1}};
\draw[very thick] (0,4.5)--(9,4.5)--(9,5.5)--(12,5.5);

% forma de onda de d2
\node[anchor=east] (d2) at (0,3.5) {$d_2$};
\node at ($(d2)+(10,0)$) {\textbf{0}};
\draw[very thick] (0,3)--(12,3);

% forma de onda de d1
\node[anchor=east] (d1) at (0,2) {$d_1$};
\node at ($(d1)+(10,0)$) {\textbf{1}};
\draw[very thick] (0,1.5)--(9,1.5)--(9,2.5)--(12,2.5);

% forma de onda de d0
\node[anchor=east] (d0) at (0,0.5) {$d_0$};
\node at ($(d0)+(10,0)$) {\textbf{1}};
\draw[very thick] (0,0)--(9,0)--(9,1)--(12,1);

\end{tikzpicture}
\end{center}


\parasabermais{%
O capítulo 9 do livro do Floyd apresenta mais detalhes sobre
registradores de deslocamento.
}

\newpage

\section{Memórias}

No contexto deste curso, \emph{memória} é qualquer circuito que permita o
armazenmento de informação digital.

Um registrador de armazenamento (visto na aula 12 e, em sua versão síncrona,
na aula 14) é um tipo de memória. Até mesmo um latch ou flip-flop do tipo D
é uma memória (para apenas 1 bit). Porém, é mais usual a aplicação do termo
``memória'' a um circuito digital cuja interface é similar à organização
abaixo.

\begin{center}
\begin{tikzpicture}
\tikzset{>=triangle 45}
%\draw[help lines, thick] (0,0) grid (8,8);

\stereobox{1}{0}{7}{7}
\node[yshift=0.5\baselineskip] at (5,3.5) {Memória para $2^m$ palavras};
\node[yshift=-0.5\baselineskip] at (5,3.5) {com $n$ bits em cada palavra};

\node[anchor=south] (dn1) at (2.5,8) {$d_{n-1}$};
\draw[<->] (dn1) -- ($(dn1.south)+(0,-1)$);

\node[anchor=south] (dn2) at ($(dn1.south)+(1,0)$) {$d_{n-2}$};
\draw[<->] (dn2) -- ($(dn2.south)+(0,-1)$);

\node[anchor=south] (ddots) at ($(dn2.south)+(1,0)$) {$\ldots$};
\node at ($(ddots.south)+(0,-0.5)$) {$\ldots$};

\node[anchor=south] (d1) at ($(ddots.south)+(1,0)$) {$d_{1}$};
\draw[<->] (d1) -- ($(d1.south)+(0,-1)$);

\node[anchor=south] (d0) at ($(d1.south)+(1,0)$) {$d_{0}$};
\draw[<->] (d0) -- ($(d0.south)+(0,-1)$);

\node[anchor=north] at (4.5,6.8) {linhas de dado};

\node[anchor=east] (a0) at (0,6.5) {$a_{0}$};
\draw[->] (a0) -- ($(a0.east)+(1,0)$);

\node[anchor=east] (a1) at ($(a0.east)+(0,-0.75)$) {$a_{1}$};
\draw[->] (a1) -- ($(a1.east)+(1,0)$);

\node[anchor=east] (adots) at ($(a0.east)+(0,-1.4444)$) {$\vdots$};
\node at ($(adots.east)+(0.5,0)$) {$\vdots$};

\node[anchor=east] (am2) at ($(a0.east)+(0,-2.25)$) {$a_{m-2}$};
\draw[->] (am2) -- ($(am2.east)+(1,0)$);

\node[anchor=east] (am1) at ($(a0.east)+(0,-3)$) {$a_{m-1}$};
\draw[->] (am1) -- ($(am1.east)+(1,0)$);

\node[anchor=north,rotate=90] at (1,4.8) {linhas de endereço};

\node[anchor=east] (c0) at (0,1.75) {$c_{0}$};
\draw[->] (c0) -- ($(c0.east)+(1,0)$);

\node[anchor=east] (cdots) at ($(c0.east)+(0,-0.6)$) {$\vdots$};
\node at ($(cdots.east)+(0.5,0)$) {$\vdots$};

\node[anchor=east] (ck1) at ($(c0.east)+(0,-1.25)$) {$c_{k-1}$};
\draw[->] (ck1) -- ($(ck1.east)+(1,0)$);

\node[anchor=north,rotate=90] at (1,1) {linhas de};
\node[anchor=north,rotate=90,yshift=-\baselineskip] at (1,1) {controle};

\end{tikzpicture}
\end{center}

\emph{Exemplo:} a Figura~\ref{fig:circuitomemoria} mostra um circuito interno
de uma memória para $4$ palavras de $n$ bits que foi construída usando-se
registradores síncronos de $n$ bits, um mux $4\times1$ de $n$ bits
e um decodificador $2\times4$.

A memória da Figura~\ref{fig:circuitomemoria} possui linhas separadas para
entrada e saída de dados (entrada: $d_{n-1}, \ldots, d_0$;
saída: $D_{n-1}$, \ldots, $D_0$), duas linhas de endereço ($a_1, a_0$) que
selecionam qual registrador será lido ou escrito, e duas linhas de controle
(\emph{clock} $C_k$ e bit de operação $Op$, que indica se um registrador será
lido quando $Op=0$, ou escrito quando $Op=1$).

A memória deste exemplo é caracterizada como uma memória \emph{estática},
\emph{síncrona}, \emph{volátil} e de \emph{leitura/escrita}.

\begin{itemize}
\item \emph{Estática:} uma vez que um dado é armazenado em uma posição
      da memória, ele é mantido até que uma operação de escrita o
      apague, ou até que a alimentação elétrica seja desligada.
\item \emph{Síncrona:} as operações na memória (neste caso, apenas a
      escrita) são sincronizadas por meio de um sinal de clock externo.
\item \emph{Volátil:} os dados armazenados serão perdidos caso a alimentação
      externa seja desligada.
\item \emph{De leitura/escrita:} são permitidas as operações de leitura e
      de escrita.
\end{itemize}

\begin{figure}
\begin{center}
\tikzstyle{branch}=[fill,shape=circle,minimum size=3pt,inner sep=0pt]
\def\buswidth#1{\phantom{$#1$} \raisebox{0.5ex}{\rule{1em}{0.4pt}} $#1$}
\begin{tikzpicture}[y=2\baselineskip,x=2\baselineskip]
\tikzset{>=stealth}
%\foreach \i in {0,...,16}
%{
%  \draw[thick,color=gray] (0,-\i) -- (16,-\i);
%  \draw[thick,color=gray] (\i,0) -- (\i,-16);
%}

\node (din) at (7.5,0) {$d_{n-1} \ldots d_0$};
\node[left=0pt of din,anchor=east] {entrada de dados:};
\coordinate (databranch) at (7.5,-1);
\draw[line width=1.5mm] (din) edge node {\buswidth{n}} (databranch);

\foreach \i in {0,1,2,3}
{
  \stereobox[\hspace*{1em}R\i]{11-3*\i}{-3}{2}{1.2}

  \coordinate (inr\i) at (12-3*\i,-1.8);
  \draw[->,line width=1.5mm](databranch)-|(inr\i);

  \cknode{ckr\i}{11-3*\i}{-2.5};
  \node[thick,and gate US, draw, logic gate inputs=nnn,rotate=90,anchor=output] (and\i) at ($(ckr\i.west)+(-0.5,-0.75)$) {};
  \draw (and\i) |- (ckr\i.west);
  \ifthenelse{\i=0}
  {
    \coordinate (ckbrch\i) at ($(and\i.input 1)+(0,-0.5)$) {};
  }
  {
    \node[branch] (ckbrch\i) at ($(and\i.input 1)+(0,-0.5)$) {};
  }

  \coordinate (outr\i) at (12-3*\i,-3);
}
\node[anchor=east] (ck) at (ckbrch3 -| 0,0) {$Ck$};
\draw (ck) -| (and0.input 1);
\draw (ck) -| (and1.input 1);
\draw (ck) -| (and2.input 1);
\draw (ck) -| (and3.input 1);

\node[anchor=east,below=0pt of ck] (op) {$Op$};
\node[branch] (opbrch3) at (op -| and3.input 2) {};
\node[branch] (opbrch2) at (op -| and2.input 2) {};
\node[branch] (opbrch1) at (op -| and1.input 2) {};
\coordinate (opbrch0) at (op -| and0.input 2) {};
\draw (op) -| (and0.input 2);
\draw (op) -| (and1.input 2);
\draw (op) -| (and2.input 2);
\draw (op) -| (and3.input 2);

% decoder
\node[anchor=east,below=3\baselineskip of op] (a1) {$a_1$};
\node[anchor=east,below=0pt of a1] (a0) {$a_0$};
\node[anchor=west] (A1) at (a1 -| 1.5,0) {$A_1$};
\node[anchor=west] (A0) at (a0 -| 1.5,0) {$A_0$};
\node[anchor=east] (X3) at ($(a1.east)+(4.2,0.6)$) {$X_3$};
\node[anchor=east,below=0pt of X3] (X2) {$X_2$};
\node[anchor=east,below=0pt of X2] (X1) {$X_1$};
\node[anchor=east,below=0pt of X1] (X0) {$X_0$};
\draw[thick] (X0.south -| 1.5,0) rectangle (X3.north east)
             node[pos=.5,yshift=0.5\baselineskip] {decod.}
             node[pos=.5,yshift=-0.5\baselineskip] {$2\times4$};

% ligações barr. endereço e decoder
\draw (a1)--(A1);
\node[branch] (a1brch) at ($(a1)!0.75!(A1)$) {};
\draw (a0)--(A0);
\node[branch] (a0brch) at ($(a0)!0.5!(A0)$) {};

% ligações decoder e portas and
\draw (X3)-|($(X3)+(0.6,0.6)$)-|(and3.input 3);
\draw (X2)-|(and2.input 3);
\draw (X1)-|(and1.input 3);
\draw (X0)-|(and0.input 3);

% mux
\coordinate (muxtopleftcorner) at ($(X0.south east)+(-0.57,-1)$);
\coordinate (muxbottomleftcorner) at ($(muxtopleftcorner)+(0.5,-1)$);
\coordinate (muxtoprightcorner) at ($(muxtopleftcorner)+(7,0)$);
\coordinate (muxout) at ($(muxtopleftcorner)!0.5!(muxtoprightcorner)+(0,-1)$);
% front face of the mux
\draw[thick,line join=round]
     (muxtopleftcorner) -- (muxbottomleftcorner)
     -- ++($(muxtoprightcorner)-(muxtopleftcorner)-(1,0)$)
     -- (muxtoprightcorner) -- (muxtopleftcorner);
\node[anchor=south] at (muxout) {\hspace*{1em}MUX $n$ bits};
% extrusion
\draw[thick,line join=round]
     (muxtopleftcorner) -- ++(0.4,0.4) --
     ++($(muxtoprightcorner)-(muxtopleftcorner)$) --
     ++(-0.4,-0.4) -- ++(0.4,0.4) -- ++(-0.5,-1)
     -- ++(-0.4,-0.4);

\node[anchor=west] (S1)
     at ($(muxtopleftcorner)!0.25!(muxbottomleftcorner)$) {$S_1$};
\draw (a1brch)|-(S1.west);

\node[anchor=west] (S0)
     at ($(muxtopleftcorner)!0.75!(muxbottomleftcorner)$) {$S_0$};
\draw (a0brch)|-(S0.west);

\node[anchor=north,yshift=-0.2\baselineskip] (I3)
     at ($(muxtopleftcorner)!0.3!(muxtoprightcorner)+(0,0.2)$) {$I_3$};

\node[anchor=north,yshift=-0.2\baselineskip] (I2)
     at ($(muxtopleftcorner)!0.475!(muxtoprightcorner)+(0,0.2)$) {$I_2$};

\node[anchor=north,yshift=-0.2\baselineskip] (I1)
     at ($(muxtopleftcorner)!0.650!(muxtoprightcorner)+(0,0.2)$) {$I_1$};

\node[anchor=north,yshift=-0.2\baselineskip] (I0)
     at ($(muxtopleftcorner)!0.825!(muxtoprightcorner)+(0,0.2)$) {$I_0$};

\draw[->,line width=1.5mm] (outr3)
                           -- node [pos=0,yshift=-\baselineskip] {\buswidth{n}}
                              ++(0,-2.8)
                           -| (I3);
\draw[->,line width=1.5mm] (outr2)
                           -- node [pos=0,yshift=-\baselineskip] {\buswidth{n}}
                              ++(0,-1)
                           -| (I2);
\draw[->,line width=1.5mm] (outr1)
                           -- node [pos=0,yshift=-\baselineskip] {\buswidth{n}}
                              ++(0,-1)
                           -| (I1);
\draw[->,line width=1.5mm] (outr0)
                           -- node [pos=0,yshift=-\baselineskip] {\buswidth{n}}
                              ++(0,-6.2)
                           -| (I0);

\draw[->,line width=1.5mm] (muxout) -- node [pos=0.4] {\buswidth{$n$}}
                           ++(0,-1) node[anchor=north] (dout)
                                    {$D_{n-1}\ldots D_0$};

\node[left=0pt of dout,anchor=east] {saída de dados:};

\end{tikzpicture}
\caption{Memória para $4$ palavras de $n$ bits cada uma.}
\label{fig:circuitomemoria}
\end{center}
\end{figure}

Contraste com o extremo oposto:
\begin{itemize}
\item \emph{Dinâmica:} após a escrita do dado, ele será ``esquecido''
      (perdido) após um certo tempo, mesmo que a alimentação elétrica
      seja mantida. Para que os dados não se percam, é necessário que
      a reescrita dos dados (\emph{refresh}) seja feita periodicamente.
\item \emph{Assíncrona:} operações na memória não dependem de um \emph{clock}
      externo.
\item \emph{Não-Volátil:} os dados são mantidos mesmo que a alimentação
      elétrica seja desligada.
\item \emph{Apenas de leitura (ROM = Read-Only Memory):} somente é permitida
      a leitura de dados pré-gravados.
\end{itemize}

\subsection{Classificação de memórias}

\begin{itemize}
\item Quanto à necessidade de \emph{refresh}:\\
      Estática (não necessita de \emph{refresh}) $\times$ Dinâmica (necessita)

\item Quanto à volatilidade dos dados:\\
      Volátil (dados perdidos após desligamento da energia) $\times$ Não-volátil

\item Quanto ao sincronismo:\\
      Síncrona (necessita de \emph{clock} externo) $\times$ assíncrona

\item Quanto à possibilidade de reescrita/apagamento:
\begin{itemize}
  \item Somente leitura (ROM = Read-Only Memory)
  \item WORM = Write Once, Read Many
  \item Regravável (para escrever, é preciso apagar a memória inteira primeiro)
  \item De leitura/escrita (R/W)
\end{itemize}

\item Quanto ao método de aceso:
\begin{itemize}
  \item aleatório ou imediato (RAM = Random Access Memory):
        cada endereço pode ser acessado (lido ou escrito) imediatamente
        após a requisição.
  \item sequencial: para ler ou escrever em um determinado endereço, é
        necessário acessar primeiramente todos os endereços da memória que
        o antecedem.
  \item direto: para um dado um endereço $A$, acessa-se de modo
        imediato uma \emph{vizinhança} desse endereço e, em seguida, é
        feita uma busca sequencial dentro da vizinhança para se encontrar
        o endereço $A$.
  \item associativo: o acesso aos dados é feito não por endereço, mas
        por comparação com parte do seu conteúdo.
\end{itemize}

\item Quanto à tecnologia: semicondutora (transistores), magnética, ótica,
      magneto-ótica, eletromecânica (relés, memória de linha de retardo de
      mercúrio, delay line memory), etc.
\end{itemize}

\subsection{Exemplos de memórias}

\begin{itemize}
\item Disco rígido (HD): estática, não-volátil, leitura/escrita, acesso direto,
      magnética.
\item Pen-drive: estática, não-volátil, leitura/escrita\footnote%
{%
  Sobre a classificação leitura/escrita para pen-drives: os dispositivos
  conhecidos atualmente com esse nome são feitos de um tipo de memória
  não-volátil chamado \emph{flash}, que é regravável, e mais uma porção de
  memória RAM volátil, porém de leitura/escrita. A memória \emph{flash} é
  um tipo de memória bem peculiar, organizada em ``células'' regraváveis.
  Para que uma célula possa ser reescrita, ela precisa precisa ser
  primeiramente apagada por meio de um pulso elétrico de grande intensidade
  e curta duração (daí o nome ``\emph{flash}'').
},%
      acesso direto, semicondutora.
\item Fita de backup: não-volátil, leitura/escrita, acesso sequencial,
      magnética.
\item CD/DVD/Blu-ray: não-volátil, ROM, acesso direto, óptica.
\item CD-ROM/DVD-ROM/BD-ROM: não-volátil, WORM, acesso direto, óptica.
\item CD-RW/DVD-RW/BD-RW: não-volátil, regravável, acesso direto, óptica.
\item DIMM SDRAM (vulgarmente chamada de ``memória RAM de PC''):
      dinâmica, síncrona, volátil, leitura/escrita, acesso imediato (RAM)
      semicondutora.
\item memória cache: (tipicamente) estática, síncrona, volátil,
      leitura/escrita, acesso associativo, semicondutora.
\end{itemize}

\parasabermais{%
O capítulo 10 do livro do Floyd apresenta mais detalhes sobre
memórias.
}

\section*{Referência}

FLOYD, Thomas L. \emph{Sistemas Digitais: Fundamentos e Aplicações}. $9^a$
edição. Bookman, 2007.


\end{document}
