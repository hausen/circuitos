\documentclass[a4paper,12pt]{article}
\usepackage{graphicx}
\usepackage[left=1.5cm,top=1.5cm,bottom=1.5cm,right=1.5cm]{geometry}
\usepackage[utf8]{inputenc}
\usepackage[T1]{fontenc}
\usepackage{lmodern}
\usepackage[portuguese,brazil]{babel}

\usepackage{amsmath}
\usepackage{slashbox}
\usepackage{array}
\usepackage{icomma} % para vírgula decimal / decimal comma
\usepackage{enumerate}

\newcounter{questao}
\setcounter{questao}{0}
\newcommand{\questao}{%
\vspace{12pt}%
\refstepcounter{questao}%
\noindent%
\textbf{Exercício \arabic{questao}.}%
{ }%
}

\begin{document}
\begin{center}
\Large{Circuitos Digitais -- Aula 6 -- Exercícios}
\end{center}

\questao Faça as conversões de base pedidas, mostrando as divisões/multiplicações efetuadas, caso sejam necessárias.

(a) $(57,55)_{10}$ p/ binário \,\;\;\;\;\; (b) $(1011,1101)_2$ p/ decimal
\;\;\;\;\;\;\; (c) $(1011,1101)_2$ p/ base 16

(d) $(1011,1101)_2$ p/ base $8$ \; (e) $($DEAD$,$BEEF$)_{16}$ p/ binário
\; (f) $($DEAD$,$BEEF$)_{16}$ p/ decimal

\questao Efetue as operações abaixo em binário. Para as subtrações, faça as contas duas vezes: usando o algoritmo padrão de subtração (com ``empréstimos'') e fazendo a conta usando complemento a 2.

(a) $11010,1 + 10110,01 + 111,1110$ \;\;\;\;
(b) $10101 - 1110$ \;\;\;\;
(c) $1011001 - 1100011$

(d) $111,10 \times 1,00101$ \hspace{2.9cm} (e) $10010101 \div 1001$

\def\bit{\fbox{\vbox to 0.6cm{}\hspace*{0.5cm}}}

\questao Converta os números abaixo para a base $2$ e represente-os no formato pedido.

\begin{enumerate}[(a)]
\item $0$ (zero) como inteiro sem sinal em uma palavra de $8$ bits

\bit\bit\bit\bit\bit\bit\bit\bit

\item $+0$ (zero positivo) como inteiro com sinal, em sinal-magnitude, em uma palavra de $8$ bits

\bit\bit\bit\bit\bit\bit\bit\bit

\item $-0$ (zero negativo) como inteiro com sinal, em sinal-magnitude, em uma palavra de $8$ bits

\bit\bit\bit\bit\bit\bit\bit\bit

\item $+0$ (zero positivo) como inteiro com sinal, em complemento de dois, em uma palavra de $8$ bits

\bit\bit\bit\bit\bit\bit\bit\bit

\item $-0$ (zero negativo) como inteiro com sinal, em complemento de dois, em uma palavra de $8$ bits

\bit\bit\bit\bit\bit\bit\bit\bit

\item $117$ como inteiro sem sinal em uma palavra de $8$ bits

\bit\bit\bit\bit\bit\bit\bit\bit

\item $+117$ como inteiro com sinal, em sinal-magnitude, em uma palavra de $8$ bits

\bit\bit\bit\bit\bit\bit\bit\bit

\item $-117$ como inteiro com sinal, em sinal-magnitude, em uma palavra de $8$ bits

\bit\bit\bit\bit\bit\bit\bit\bit

\item $+117$ como inteiro com sinal, em complemento de dois, em uma palavra de $8$ bits

\bit\bit\bit\bit\bit\bit\bit\bit

\item $-117$ como inteiro com sinal, em complemento a dois, em uma palavra de $8$ bits

\bit\bit\bit\bit\bit\bit\bit\bit

\item $175$ como inteiro sem sinal em uma palavra de $8$ bits.

\bit\bit\bit\bit\bit\bit\bit\bit

\item $+175$ como inteiro com sinal, em sinal-magnitude, em uma palavra de $8$ bits

\bit\bit\bit\bit\bit\bit\bit\bit

\item $-175$ como inteiro com sinal, em sinal-magnitude, em uma palavra de $8$ bits

\bit\bit\bit\bit\bit\bit\bit\bit

\item\label{plus175} $+175$ como inteiro com sinal, em complemento de dois, em uma palavra de $8$ bits

\bit\bit\bit\bit\bit\bit\bit\bit

\item\label{minus175} $-175$ como inteiro com sinal, em complemento a dois, em uma palavra de $8$ bits

\bit\bit\bit\bit\bit\bit\bit\bit

\end{enumerate}

\questao Converta as palavras de dados nos itens (\ref{plus175}) e
(\ref{minus175}) de volta para um numeral na base decimal. Por que
o valor obtido é diferente do original?


\questao Considere duas variáveis lógicas, $C$ -- que indica se \textbf{chove} -- e $F$ -- que indica se faz \textbf{frio}, e as funções lógicas abaixo:

\begin{itemize}
\item $P$ -- o tempo está \textbf{péssimo} quando \textbf{chove e faz frio};
\item $R$ -- o tempo está \textbf{ruim} quando \textbf{chove ou faz frio};
\item $M$ -- o tempo está \textbf{mais ou menos} quando \textbf{chove mas não faz frio, ou vice-versa};
\item $B$ -- o tempo está \textbf{bom} quando \textbf{não chove nem está frio};
\item $S$ -- o tempo está \textbf{seco} quando \textbf{não chove}.
\end{itemize}

(a) Complete as tabelas verdade abaixo, onde $1$ representa \emph{verdadeiro} e
$0$ representa \emph{falso}.

\begin{center}
\begin{tabular}{|cc||c|c|c|c|c|}
\hline
$C$ & $F$ & $P$ & $R$ & $M$ & $B$ & $S$ \\
\hline
$0$ & $0$ &     &     &     &     &     \\
$0$ & $1$ &     &     &     &     &     \\
$1$ & $0$ &     &     &     &     &     \\
$1$ & $1$ &     &     &     &     &     \\
\hline
\end{tabular}
\end{center}

(b) Deduza expressões lógicas para as funções.

(c) Qual o contrário de $R$ -- tempo ruim? E de $M$ -- tempo mais ou menos?

\questao Escreva a tabela verdade de cada uma das expressões abaixo e
represente-as na forma padrão de soma-de-produtos.\\

\hfill (a) $X + \overline{Y + Z}$ \hfill (b) $\overline{X}(Y + Z) + XY$
\hfill (c) $\overline{\overline{X + Y} + Z}$ \hspace*{\stretch{1}}

\questao Construa o mapa de Karnaugh de:\\

\hfill (a) $F(A,B,C) = AB + \overline{B}C + AC$
\hfill (b) $Z = \overline{A}BD + BC\overline{D} + \overline{B} \, \overline{C} D + A \overline{B} \, \overline{D}$ \hspace*{\stretch{1}}

\questao Escreva na forma mínima de soma-de-produtos:
$$\overline{A} \, \overline{B} C D E + 
 \overline{A} B \overline{C} \, \overline{D} E +
 A B \overline{C} D E +
 \overline{A} B \overline{C} D E +
 \overline{A} B C D E +
 A B \overline{C} \, \overline{D} E +
 (\overline{A + B}) E$$

\end{document}

